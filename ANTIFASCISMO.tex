\begin{quote}




O fascismo é uma palavra de oito letras que começa pela letra f. O
homem, desde sempre, foi perdidamente apaixonado pelos jogos de palavras
que, escondendo a realidade mais ou menos bem, absolvem-no da reflexão
pessoal e da decisão. Assim o símbolo age em nosso lugar e nos fornece
um álibi e uma bandeira.

Quanto ao símbolo com o qual não pretendemos nos aliar, que na verdade
nos enoja profundamente, aplicamo-lo a palavra ``anti'', consideramo-nos
do outro lado, seguros, e pensamos haver nos livrado com isso de uma boa
parte das nossas tarefas. Assim, uma vez que na mente de muitos de nós,
e quem escreve se encontra entre estes, o fascismo causa nojo, é
suficiente o recurso daquele ``anti'' para sentirmos nossa consciência
limpa, encerrados em um campo bem guardado e bem frequentado.

Entretanto a realidade se move, os anos passam e as relações de força se
modificam. Novos mestres sucedem aos antigos e o trágico bastão do poder
passa de mão em mão. Os fascistas de ontem puseram de lado as bandeiras
e as suásticas, entregues a alguns estúpidos de grandes carecas, e se
adequaram ao jogo democrático. Por que não fariam isso? Os homens do
poder são unicamente homens do poder, a balela nasce e morre, o realismo
político não. Mas nós, que de política compreendemos pouco ou nada,
indagamo-nos envergonhados pelo ocorrido, visto que nos foi retirado,
bem debaixo de nossos narizes, as antigas evidências do fascista da
camisa negra\footnote{Milícia Voluntária para a Segurança Nacional que apoiou Mussolini e o
  ajudou a dar o seu golpe de Estado, na Itália. Também estiveram
  presentes na Marcha sobre Roma, dia conhecido como o fim da democracia
  liberal italiana e a ascensão do fascismo. Essa milícia que combatia
  com violência grevistas, sindicatos e opositores do fascismo, passou a
  integrar oficialmente o exército italiano, durante o período em que
  Mussolini esteve no governo. Inspirou diversos movimentos na Europa e
  nas Américas com seus devidos uniformes prata, azul, verde etc. e até
  hoje serve de inspiração para saudosistas.} com um taco, contra os quais éramos acostumados a lutar
muito duramente. Por isso vamos à caça como galinhas sem cabeça, por um
novo bode expiatório contra o qual descarregar nosso ódio demasiado
barato, enquanto tudo em torno de nós fica mais sutil e mais esfumaçado,
enquanto o poder nos chama para discutir:

--- Mas por favor, venha para frente, diga a sua opinião, sem embaraço!
Não esqueça, estamos em uma democracia, cada um tem direitos de falar
quanto e como quiser. Os outros o escutam, concordam ou discordam e
depois o número faz o jogo final. A maioria vence e à minoria resta o
direito de voltar a discordar. Contanto que tudo se mantenha na livre
dialética de tomar os lados.

Se nós colocamos a questão do fascismo sob os termos da balela, devemos
forçosamente admitir que tudo não passou de um jogo. Talvez, uma ilusão:

--- O Mussolini, um bravo homem, de certo um grande político. Cometeu
seus erros. Mas quem não os comete? Então, deixou-se levar. Ele foi
traído. Todos fomos traídos. A mitologia fascista e antigo-romana? Deixa
disso! Ela pensa agora esse antiquário? Rouba do passado.

``Hitler... --- ironizava Klaus Mann\footnote{Escritor alemão, conhecido pelo seu romance \emph{Mephisto}, obra que
  questiona o nazismo na Alemanha, abordando a história de um ator que
  interpretava o papel de Mefisto na peça \emph{Fausto}, de Goethe, uma
  das poucas peças interpretadas durante o período. Mann era gay e fez
  parte do grupo de teatro \emph{die Pfeffermühle}, de sua irmã, Erika
  Mann, que ridicularizava os nazistas. Deixou a Alemanha com a família,
  em 1933, unindo-se a seu pai, Thomas Mann.} descrevendo muito bem a mentalidade de Gerhart
Hauptmann,\footnote{Romancista e dramaturgo alemão, ganhou o Nobel de Literatura de 1912,
  foi um dos responsáveis pela introdução do naturalismo no teatro
  alemão. Suas obras, com o decorrer do tempo, formaram um complexo
  metafísico e religioso. Era um dos artistas da \emph{lista de
  Gottbegnadeten}, uma lista elaborada por Goebbels e Hitler e continha
  os nomes de todos os artistas cruciais para a cultura nazista e que
  seriam automaticamente dispensados do serviço de guerra.} o velho teórico do realismo político --- no fim das
contas... Meus caros amigos!... Sem malícia!... Procuremos ser... Não,
se não vos importa, permitam-me... objetivos... Posso encher de novo a
taça? Esta champagne... extraordinária, realmente --- o homem Hitler,
como ia dizendo... Também a champagne, quanto a isto... Uma evolução
absolutamente extraordinária... A juventude alemã... Cerca de 7 milhões
de votos... Como disse frequentemente para meus amigos judeus... Aqueles
alemães... nação incalculável... realmente misteriosíssima... impulsos
cósmicos... Goethe... A saga dos Nibelungos...\footnote{Lendas de povos bárbaros que
  passaram por ``interpretações'' diversas ao longo da história, como a
  \emph{Saga dos volsungos} (em língua escandinava), a \emph{Canção dos
  nibelungos} (em língua alemã) e \emph{O Anel do Nibelungo} (na música
  alemã). Por meio desta, por exemplo, Richard Wagner, seu compositor,
  exaltará o povo alemão e será tomado pelo sentimento moral e cristão.
  Também a partir desta que Nietzsche se afastará de Wagner, em 1876:
  ``Eu não tolero nada ambíguo; depois que Wagner mudou-se para a
  Alemanha, ele transigiu passo a passo com tudo o que desprezo --- até
  mesmo o antissemitismo... Era de fato o momento para dizer adeus: logo
  tive a prova disto. Richard Wagner, aparentemente o mais triufante, na
  verdade um \emph{décadent} desesperado e fenecido, sucumbiu de
  repente, desamparado e alquebrado, ante a cruz cristã...'' (Nietzsche,
  \emph{O caso Wagner}).} Hitler, de um certo modo, exprime... Como eu procurei
explicar aos meus amigos judeus... tendências dinâmicas... elementares,
irresistíveis...''.

Não, sob os termos da balela não. De frente a uma boa taça de vinho, a
diferença se esfumaça e tudo se torna questão de opinião, discutível. O
belo é isso: a diferença existe, não entre fascismo e antifascismo, mas
entre quem quer e querendo-o persegue e gere o poder e quem o combate e
o refuta. Mas sob quais termos podemos encontrar um fundamento concreto
para esta diferença?

Talvez sob os termos de uma análise mais profunda? Talvez fazendo
recurso a uma análise histórica?

Não creio. Os historiadores constituem a mais útil categoria de imbecis
a serviço do poder. Acreditam saber muitas coisas, mas mais se obstinam
sobre o documento, mais não fazem outra coisa senão sublinhar a
necessidade de ser como tal, um documento que atesta de modo
incontestável o ocorrido, a prisão da vontade do indivíduo na
racionalidade do dado, a equivalência vichiana\footnote{Da filosofia de Giambattista Vico (1668 -- 1744), filósofo,
  historiador e jurista italiano.} da verdade e do fato. Cada consideração sobre possíveis
eventualidades ``outras'' redunda em simples passatempo literário. Cada
inferência, capricho absurdo. Quando o historiador tem um lampejo de
inteligência, excede imediatamente em outro lugar, nas considerações
filosóficas, e aí cai na área comum a esse gênero de reflexão. Contos de
fadas, gnomos e castelos encantados. E enquanto isso tudo ao redor do
mundo se ajusta nas mãos dos poderosos que fizeram a própria cultura dos
caderninhos de revisão, que não distinguiriam um documento de uma batata
frita. ``Se a vontade de um homem fosse livre, escreve Tolstói em
\emph{Guerra e Paz}, toda a história seria uma série de fatos
fortuitos... Se, ao invés, existe apenas uma lei que governa as ações
dos homens, não pode existir a liberdade do arbítrio, uma vez que a
vontade dos homens deve estar sujeita a esta lei''.

O fato é que os historiadores são úteis, sobretudo para fornecer
elementos de conforto. Álibis e muletas psicológicas. Quão bons foram os
federados da Comuna de 1871! Como são corajosos os mortos em Père
Lachaise!\footnote{O maior cemitério de Paris.} E o leitor inflama-se e prepara-se ele também para
morrer, se necessário, no próximo muro dos federados. Nessa espera, isto
é, à espera de que as forças sociais objetivas nos coloquem em condições
de morrer como heróis, nos arranjem a vida de todos os dias, para depois
chegarmos à soleira da morte antes que essa tão esperada ocasião nos
tenha estado à porta. As tendências históricas não são tão exatas,
década a mais, década a menos, podemos saltá-las e retornar com nada nas
mãos.

Se quer mensurar a imbecilidade de um historiador, leve-o a raciocinar
sobre a coisa em andamento e não sobre o passado. Vai ouvir algumas
belezuras.

Não, a análise histórica não. Talvez a análise política, ou
político-filosófica, como estivemos habituados a ler nesses últimos
anos. O fascismo é isso, e depois aquilo e agora isso e aquilo. A
técnica de formulação dessas análises é a seguinte. Toma-se o mecanismo
hegeliano de dizer e contradizer ao mesmo tempo, qualquer coisa similar
à crítica das armas que se torna as armas da crítica,\footnote{Referência ao ditado marxista decorrente da introdução da
  \emph{Crítica da filosofia do direito de Hegel}: ``A arma da crítica
  não pode, é claro, substituir a crítica da arma, o poder material tem
  de ser derrubado pelo poder material, mas a teoria também se torna
  força material quando se apodera das massas''.} extraindo de uma afirmação aparentemente clara tudo
aquilo que passa pela cabeça naquele momento. Você conhece o sentimento
de desilusão quando, ao perseguir um ônibus inutilmente, percebe que o
motorista, tendo o visto, acelerou ao invés de parar? Bem, neste caso
pode-se demonstrar, e Adorno me parece que o tenha feito, que é
justamente a frustração inconsciente e remota que vem à tona, causada
pela vida que foge e que não podemos agarrar, nos impelindo a desejar
matar o motorista. Mistérios da lógica hegeliana. Desse modo,
tranquilamente, o fascismo torna-se qualquer coisa menos desprezível.
Dado que dentro de nós, no recanto obscuro do instinto bestial que faz
aumentar as pulsações, fica agachado um fascista desconhecido a si
mesmo, somos levados a justificar todos os fascistas em nome do
potencial fascista que há em nós. Certo, os extremismos não! Isso nunca.
Os pobres judeus, nos fornos! Mas foram tantos assim que morreram neles?
Seriamente, pessoas dignas do máximo respeito, em nome de um mal
compreendido senso de justiça, colocaram em circulação a estupidez de
Faurisson.\footnote{Robert Farisson (1929 -- 2018), professor na universidade de Lyon, foi
  um ``negador do Holocausto'', tornando-se conhecido pelo artigo
  publicado no \emph{Le Monde} intitulado ``O problema das câmaras de
  gás ou o rumor de Auschwitz''. Farisson ``provou'' que as câmaras de
  gás nos campos de extermínio não existiam e que o genocídio realizado
  pelos nazistas não havia acontecido. Ele, assim como outros
  ``negacionostas do Holocausto'', renegam o termo e preferem ser
  chamados de ``revisionistas''.} Não, sobre essa estrada é melhor não caminhar mais.

A raposa é inteligente e portanto tem inúmeras razões próprias e tantas
outras ainda pode conceber até dar a impressão de que o pobre
porco-espinho está sem argumentos, mas não é bem assim.

A palavra é uma arma mortal. Escava por dentro o coração do homem e lhe
insinua a dúvida. Quando o conhecimento é escasso e aquelas poucas
noções que possuímos parecem dançar em um mar em tempestade, caímos
facilmente nas presas dos equívocos gerados por aqueles que são melhores
do que nós com as palavras. Para evitar casos do gênero, os marxistas,
como bons programadores da consciência alheia, em particular a do
proletariado arrebanhado, sugeriram a equivalência entre fascismo e
cassetete. Também filósofos de todo respeito, como Gentile,\footnote{Giovanni Gentile (1875 -- 1944) foi um filósofo, político e educador
  italiano. Autointitulado o filósofo do fascismo, forneceu base
  intelectual para o fascismo italiano, e escreveu, sob pseudônimo,
  \emph{A doutrina fascista}, com Mussolini.} do lado oposto (mas oposto até que ponto?), sugeriram
que o cassetete, agindo sobre a vontade, é também um meio ético, dado
que constrói a futura simbiose entre Estado e indivíduo, naquela Unidade
superior que almejam chegar tanto o ato isolado quanto o ato coletivo.
Aqui vemos, seja dito entre parênteses, como marxistas e fascistas
provêm da mesma estirpe idealista, com todas as consequências práticas
do caso: incluindo o campo de concentração. Mas caminhemos adiante. Não.
O fascismo não é somente cassetete e não é nem mesmo apenas Pound,
Céline, Mishima ou Cioran.\footnote{Autores que defenderam posicionamentos nazistas.} Não é nenhum de todos esses elementos nem ainda outros
tomados isoladamente, mas o conjunto disso tudo. Não é a rebelião de um
indivíduo isolado, que escolhe a sua luta pessoal contra os outros,
todos os outros, às vezes incluindo até mesmo o Estado, e a qual pode
também nos atrair por causa da simpatia humana que temos por todos os
rebeldes, também pelos desconfortáveis. Não, não é ele o fascismo. Não é
portanto que por defendermos sua revolta pessoal possamos titubear em
nossa visceral aversão ao fascismo. Na verdade, muitas vezes,
identificando-nos com essa defesa isolada, atraídos pelo caso da coragem
e do empenho individual, confundimos ainda mais nossas ideias e as
daqueles que nos ouvem, provocando inúteis tempestades em copos d'água.

As palavras nos matam se não prestarmos atenção.

Para o poder, o fascismo nu e cru, tal como se concretizou
historicamente em períodos históricos e em regimes ditatoriais, não é
mais um conceito político praticável. Novos instrumentos irrompem na
soleira da prática gestionária do poder. Deixemo-lo portanto aos dentes
aguçados dos historiadores, que o mastiguem conforme lhes aprouverem.
Também como injúria, ou acusação política, o fascismo está fora de moda.
Quando uma palavra é usada em tom depreciativo por aqueles que gerem o
poder, não podemos fazer uso igual. E como essa palavra e o relativo
conceito nos enojam, seria bom colocar um e outro no sótão dos horrores
da história e não pensar mais nisso.

Não pensar mais na palavra e no conceito, não no que este e aquela
significam mudando a roupagem lexical e a composição lógica. É sobre
isso que temos que continuar a refletir para nos prepararmos para agir.
Olhar hoje ao redor à caça do fascista pode ser um esporte prazeroso,
mas pode também esconder a inconsciente intenção de não querer caminhar
a fundo na realidade, por trás da densa teia de um tecido de poder que
se torna sempre mais complexa e difícil de interpretar.

Compreendo o antifascismo. Sou eu também um antifascista, mas os meus
motivos não são os mesmos de tantos outros que ouvi no passado e
continuo a ouvir ainda hoje, ao definirem-se antifascistas. Para muitos,
há vinte anos, o fascismo devia ser combatido onde estava o poder. Na
Espanha, em Portugal, na Grécia, no Chile, etc. Quando, nesses países,
ao velho regime fascista sucedeu o novo regime democrático, o
antifascismo de tantos ferocíssimos opositores se apagou. Naquele
momento, notei que meus velhos companheiros de estrada tinham um
antifascismo distinto do meu. Para mim não havia mudado muita coisa.
Aquilo que fazíamos na Grécia, na Espanha, nas colônias portuguesas e
nos outros países, poderia ser feito mesmo depois, mesmo quando o Estado
democrático havia levado a vantagem, herdando as glórias passadas do
velho fascismo. Mas nem todos estavam de acordo.

Compreendo o velho antifascista, a ``resistência'', as memórias da
montanha e todo o resto. É preciso saber escutar os velhos companheiros
que recordam suas aventuras, tragédias, os tantos mortos assassinados
pelos fascistas, a violência e todo o resto. ``Mas, ainda disse Tolstói,
o indivíduo que desempenha um papel nos acontecimentos históricos nunca
compreende seu significado. Se tenta compreendê-lo, torna-se um elemento
estéril''. Compreendo menos aqueles que, sem terem vivido essas
experiências e portanto sem se verem forçosamente prisioneiros daquelas
emoções de meio século atrás, emprestam explicações que não têm razão de
ser e que frequentemente constituem uma simples fachada para se
qualificarem.

--- Eu sou antifascista! Eles jogam na minha cara a afirmação como uma
declaração de guerra, e você?

Neste caso me vem quase sempre a espontânea resposta --- Não, eu não sou
antifascista. Não sou antifascista do mesmo modo que você. Não sou
antifascista porque combatia os fascistas em seus territórios enquanto
você estava aconchegado no calor da democrática nação italiana, a qual
nem por isso deixava de eleger o governo dos mafiosos do Scelba, do
Andreotti e do Cossiga.\footnote{Mario Scelba, Giulio Andreotti e Francesco Cossiga, políticos da
  Democracia Cristã, partido italiano fundado em 1943, por Alcide De
  Gasperi. Junto a Robert Schuman, um dos fundadores do Movimento
  Republicano Popular, partido democrata cristão francês, e Konrad
  Adenauer, presidente da União Democrata-Cristã, da Alemanha, Alcide
  fundou a Comunidade Europeia do Carvão e do Aço (CECA), a qual
  desembocou na União Europeia e que lhes rendeu o título de ``pais da
  Europa''. Visavam reconstruir uma Europa pacífica legitimando o Estado
  pós-guerra por meio da garantia do exercício das liberdades
  econômicas. O Partido Social-Democrata da Alemanha só aderirá a essa
  alternativa ao capitalismo na década de 1950, renunciando ao projeto
  de socialização dos meios de produção e defendendo o direito do Estado
  de proteger a propriedade privada.} Não sou antifascista porque continuei a combater a
democracia que havia substituído aqueles fascismos de novela, empregando
meios de repressão mais modernos e, por conseguinte, se quisermos, mais
fascistas do que o fascismo que lhe havia precedido. Não sou
antifascista porque ainda hoje procuro identificar o atual detentor do
poder e não me deixo deslumbrar pelo rótulo e pelo símbolo, enquanto
você continua a dizer-se antifascista para ter justificativa para ir às
ruas esconder-se por trás de uma faixa onde está escrito ``abaixo o
fascismo!''. Claro, se eu tivesse mais que meus oito anos à época da
``resistência'', talvez eu também estaria agora abatido pelas memórias e
antigas paixões jovens e não seria tão lúcido. Mas penso que não.
Porque, se bem se escrutinam os fatos, mesmo no amontoado confuso e
anônimo do antifascismo de formação política, havia aqueles que não se
adequavam, que se excediam, que continuavam, que insistiam indo muito
além do ``cessar fogo!''. Porque a luta, pela vida e pela morte, não se
dá somente contra o fascista de ontem ou de hoje, aquele que veste a
camisa negra, mas também e fundamentalmente contra o poder que nos
oprime, com todas as suas estruturas de sustento que o tornam possível,
também quando este poder se veste de hábitos permissíveis e tolerantes
da democracia.

--- Mas agora, podem dizer subitamente --- qualquer um poderia replicar,
pegando-me desprevenido, --- também é você um antifascista! E como
poderia ser de outro modo? É um anarquista, logo, é antifascista! Não
nos canse com as suas distinções.

E, ao invés, eu penso ser útil distinguir. O fascista jamais me agradou
e, consequentemente, o fascismo como projeto, por outros motivos, os
quais, quando aprofundados, resultam nos mesmos motivos pelos quais
jamais me agradaram o democrático, o liberal, o republicano, o
gaullista, o trabalhista, o marxista, o comunista, o socialista e todos
os outros. Contra eles, tenho oposto não tanto o meu ser anárquico, mas
o meu ser distinto e portanto anárquico. Primeiro de tudo, a minha
distinção individual, o meu modo pessoal, meu e de nenhum outro, de
entender a vida, de compreendê-la e, por conseguinte, de vivê-la, de
provar as emoções, de pesquisar, examinar, descobrir, experimentar,
amar. Dentro deste meu mundo permito o ingresso unicamente daquelas
ideias e daquelas pessoas que me agradam, do resto mantenho distância,
de maneira boa ou ruim. Não me defendo, mas ataco. Não sou um pacifista
e não espero que o nível de alerta seja ultrapassado, procuro eu tomar a
iniciativa contra todos aqueles que, mesmo potencialmente, podem
constituir um perigo para o meu modo de viver a vida. E deste modo de
vida faz parte também a necessidade, o desejo dos outros. Não dos outros
enquanto entidade metafísica, mas dos outros bem identificados, daqueles
que têm afinidade com aquele meu modo de viver e ser. E esta afinidade
não é estática, selada de uma vez por todas, mas dinâmica, se modifica e
cresce, se alarga gradualmente mais e mais, chamando outras ideias e
outros homens ao seu interior, tecendo um tecido de relações imenso e
heterogêneo, onde contudo a constante que resta é sempre aquela do meu
modo de ser e de viver, com todas as suas variações e evoluções.

Atravessei em cada direção o reino dos homens e até agora não compreendi
onde poderei pousar com satisfação a minha ânsia de conhecimento, de
distinção, de paixão perturbadora, de sonho, de amante enamorado do
amor. Em toda parte vi potencialidades imensas se deixarem esmagar pela
inaptidão e pela pouca capacidade de florescerem no solo da constância e
do empenho. Mas até onde floresce a abertura para o diferente, para a
disponibilidade de penetrar e ser penetrado, até onde não há medo do
outro, mas consciência dos próprios limites e da própria capacidade,
portanto, aceitação dos limites e da capacidade do outro, há afinidade
possível, possível sonho da façanha conjunta, duradoura, eterna, para
além de uma aproximação humana contingente.

Movendo-me para fora, para territórios sempre mais distantes daquele que
descrevi, a afinidade se devanesce e desaparece. E eis os estranhos,
aqueles que portam os próprios sentimentos como decoração, aqueles que
exibindo os músculos fazem de tudo para parecerem fascinantes. E, ainda
mais longe, os sinais do poderio, os lugares e os homens do poder, da
vitalidade cumpulsiva, da idolatria que aparenta mas não é, do incêndio
que não queima, do monólogo, da balela, do ruído, do útil que tudo
mensura e tudo pesa.

É daqui que me mantenho distante e este é o meu antifascismo.