\begin{itemize}
\item \textbf{Antifa: modo de usar} reúne ensaios, textos e entrevistas de importantes pensadores contemporâneos do anarquismo e do antifascismo, como do historiador americano Mark Bray --- especialista no movimento antifascista e uma das principais vozes do momento de luta --- e dos pesquisadores e militantes Acácio Augusto, Camila Jourdan, Erick Rosas, Alfredo Bonanno e Matheus Marestoni, compondo um material urgente e essencial para nosso tempo.
A ascensão ao poder de uma direita radicalizada, nova sobretudo nos métodos de ação e no uso eficiente das tecnologias modernas, impõe uma reflexão de ordem tática: o que é, hoje, o antifascismo? Como a convulsão social pode organizar"-se contra o controle imposto por polícias ultraviolentas?
Se por um lado governantes como Jair Bolsonaro ou Donald Trump dispõem de amplo arsenal fático e bélico, a explosão recente de protestos pelo mundo aponta para uma janela de ação, em que a revolta popular também se radicaliza e, literalmente, coloca nações inteiras em chamas.


\item \textbf{Acácio Augusto} é professor no curso de Relações Internacionais da
  Universidade Federal de São Paulo (\versal{EPPEN-UNIFESP}) e coordenador do
  \versal{LASI}n\versal{T}ec/\versal{UNIFESP} (Laboratório de Análise em Segurança Internacional e
  Tecnologias de monitoramento). É pesquisador no Nu"-Sol
  (Núcleo de Sociabilidade Libertária) e autor de
  \emph{Anarquia y lucha antipolítica -- ayer y hoy} (Barcelona:
    NoLibros, 2019), \emph{Política e polícia: cuidados, controles e
  penalizações de jovens} (Rio de Janeiro: Lamparina, 2013), dentre
  outros.

\end{itemize}

