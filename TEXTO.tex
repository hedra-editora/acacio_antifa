
\chapter{Apresentação: Antifa como minoria potente}

\lipsum[5]
\lipsum[2]
\lipsum[1]

\chapter{Introdução: Fascismo e antifascismo no século \versal{XXI}}

\hfill{}\textsc{acácio augusto\footnote{Professor de Relações Internacionais da \versal{UNIFESP}, coordenador do \versal{LASI}n\versal{T}ec/\versal{UNIFESP} e pesquisador no nu"-sol/\versal{PUC"-SP}.}}

\bigskip

No momento em que o Brasil se torna o epicentro da pandemia do novo
coronavírus e os números de contaminados e mortos são cada vez mais
assustadores, soa absurdo e anacrônico, considerando o ponto de vista
histórico, falarmos em fascismo e antifascimo na segunda década de
século XXI. Mas há bons motivos para que o façamos.

O mundo assiste ao avanço inédito de partidos, movimentos e governantes
que exibem modos, símbolos e discursos fascistas, chamados de extrema
direita ou \emph{alt"-right} (do inglês \emph{alternative right}). Para
citar alguns exemplos, há a Frente Nacional Francesa, o Aurora Dourada
grego, a PEGIDA alemã, o Partido da Liberdade austríaco, o Partido Lei e
Justiça (PiS) polonês, a Liga Norte italiana e governantes ubuescos como
Viktor Orbán na Hungria, Volodymyr Zelensky na Ucrânia, Recep Tayyip
Erdoğan na Turquia, Rodrigo Duterte nas Filipinas, Jeanine Áñez na
Bolívia e Jair Bolsonaro no Brasil. Ainda que possuam trajetórias
políticas e modos de ação particulares, são lideranças que compõem um
mesmo modus operandi.

De outro lado, na Europa e nas Américas, grupos anarquistas e
antiautoritários se esforçam há mais de 40 anos em alertar que a
política (neo)fascista e (neo)nazista não foi enterrada com a derrota
das potências do Eixo na II Guerra Mundial (1939-1945), sendo, portanto,
preciso combatê-la. Esses grupos, oriundos da contracultura anarco"-punk
e do autonomismo europeu, se autodenominam \emph{antifa} ou
\emph{fantifa} (feministas antifascistas). Como define o historiador
Mark Bray, em livro publicado recentemente no Brasil, ``Antifa: o manual
antifascista'' (Autonomia Literária, 2019), o antifascismo contemporâneo
é ``um método político, um lócus de auto identificação individual e de
grupo, de um movimento transnacional que adaptou correntes socialistas,
anarquistas e comunistas preexistentes à uma súbita necessidade de
reagir à ameaça fascista'' (p. 30). O que faz da \emph{antifa}
contemporânea uma ``política nada liberal'' (p. 31) e relacionada a
movimentos de revolta e/ou revolucionários compostos de associações por
afinidade, com atuação regular e numericamente reduzidos.

Embora esses dois motivos sirvam como pano de fundo, a \emph{antifa}
virou assunto em todo mundo, em meio a pandemia da Covid"-19,
primeiramente porque o presidente estadunidense Donald Trump chamou
esses grupos de terroristas da ``esquerda radical do mal'' em um
pronunciamento no jardim da Casa Branca, no dia seguinte aos motins de
rua que se espalharam por todo o território estadunidense em protesto
contra o assassinato, por sufocamento, de George Floyd por um policial,
Eric Gardner. As manifestações de rua atacam o racismo sistemático da
sociedade e inerente à atividade policial. Remetem a outros momentos em
que um policial branco tirou a vida de uma pessoa negra, como em 2014,
no caso do assassinato do jovem Michel Brown, em Ferguson, Missouri, que
à época deu destaque ao movimento \emph{Black Lives Matter}, criado em
2013. Os atos de rua foram puxados pelo movimento negro, mas logo
ganharam adesão multitudinária.

Em seu pronunciamento, Trump ameaçou usar as das Forças Armadas,
invocando uma Lei de Insurreição de 1807, alegando que a ``nação foi
dominada por anarquistas profissionais, multidões violentas,
incendiários, saqueadores, criminosos, manifestantes, antifa e outros''.
O disse mesmo não apresentando nenhuma informação consistente de que os
saques e depredações ocorridos durante as manifestações em diversos
estados dos EUA fossem obra de anarquistas e/ou grupos \emph{antifa}. Na
esteira das declarações de Trump, como já é de costume, Jair Bolsonaro e
seus filhos, junto aos ativistas e parlamentares bolsonaristas, também
deram declarações que visam criminalizar os grupos \emph{antifas}
presentes em manifestações de rua.

No domingo, 31 de maio, e na semana seguinte, ocorreram manifestações em
diversas capitais do país, convocadas por grupos de torcidas organizadas
de times de futebol que defendem a democracia. O objetivo foi expressar
oposição às regulares manifestações de grupos bolsonaristas que
invariavelmente pedem o fechamento do STF e do Congresso Nacional,
invocam o Ato Institucional nº 5, de 1968, e conclamam o acionamento,
por parte do presidente da República, do artigo 142 da Constituição
Federal de 1988, para intervenção militar nos poderes legislativo e
judiciário. O deputado federal bolsonarista pelo Rio de Janeiro, Daniel
Silveira (PSL), ex"-policial militar, celebrizado por quebrar uma placa
ao lado do então candidato a governador Wilson Witzel (PSC), com o nome
da vereadora Marielle Franco, executada em 2018, apresentou a PL
3019/2020 para alterar a Lei 13.260/2016 (Lei Antiterrorismo). O
objetivo da PL é considerar ``organização terrorista os grupos
denominados antifas (antifascistas) e demais organizações com ideologias
similares''. Se Trump ameaçou criminalizar por terrorismo a
\emph{antifa} dos EUA, como já exigiram os grupos da \emph{alt"-right}
depois de um episódio na Universidade Berkeley durante palestra do
supremacista Milo Yiannopoulos, em 2017, o bolsonarsimo fez disso uma PL
no Congresso Nacional daqui.

Uma série de análises poderiam derivar desses episódios. Dentre elas, há
um longo e inconcluso debate no campo das Relações Internacionais e do
Direito Internacional sobre a plasticidade e o sentido político das
definições e da tipificação do ``crime'' de terrorismo, reanimado desde
os atentados de 11 de setembro de 2001. Outra possibilidade seria, a
partir da PL 3019/2020, discutir como é possível, numa democracia,
encaminhar uma proposta de criminalização de um movimento político
dissidente acompanhado do genérico ``e demais organizações similares''.
Mas o que esses acontecimentos revelam é uma questão mais urgente na
política contemporânea, não apenas no Brasil, mas em todo planeta: o
sentido do fascismo e do antifascsismo hoje.

A reação virulenta da \emph{alt"-right} global, seja de Trump e seus
apoiadores, seja do bolsonarismo no Brasil, revela uma verdadeira
obsessão pela busca de um inimigo identificável, ainda que construído
discursivamente. Como observou Mark Bray, em texto publicado
recentemente no \emph{Washington Post}, a \emph{antifa} não é o
problema. Trump a usa para distrair a opinião pública do problema
central colocado pelas manifestações nos EUA: a violência policial. No
caso do Brasil, acusar a \emph{antifa} de terrorismo é criar o inimigo
perfeito para mobilizar os aproximadamente 30\% que compõem a base
eleitoral e política do bolsonarismo. E, claro, também justificar a
violenta repressão da polícia a toda e qualquer manifestação de oposição
ao governo em curso. Não é, portanto, a presença de \emph{antifas} em
manifestações que dispara a virulência discursiva dos bolsonaristas e/ou
a violência policial, mas a necessidade de construção de um inimigo que
encarne o mal absoluto. Se a \emph{antifa} simplesmente não existisse,
criariam outro alvo, como o fizeram em novembro de 2017, quando, diante
do SESC Pompéia, tentaram impedir a realização de uma palestra da
filósofa Judith Butler sobre ``Os fins da democracia''. O ato foi
liderado pela organização Direita São Paulo, que tem como
vice"-presidente o deputado estadual de São Paulo Douglas Garcia. Ainda
que esses grupos rechacem a designação de fascistas, sua oposição tão
estridente revela alguma coisa. Como dizem os mais velhos, se tem bigode
de gato, pelo de gato e mia como um gato, um porco que não é. Ou seria?

Ainda que haja um pertinente debate entre pesquisadores das Ciências
Humanas sobre a precisão conceitual em caracterizar a \emph{alt"-right}
global como algo análogo ao fascismo histórico, é justamente o embate
nas ruas e no campo discursivo que faz da existência da \emph{antifa,} e
o crescente interesse por ela, uma chave analítica para compreender o
fascismo no século XXI. A história não se repete, ainda que alguns
discordem. Cada acontecimento possui sua singularidade na história das
lutas. Logo, ao analista cabe captar as diferenças de uma força política
a partir das metamorfoses provocadas pelas lutas para produzir um
diagnóstico do presente. Assim, de fato, se buscarmos todas as
características do fascismo histórico, dificilmente elas corresponderiam
literalmente aos modos e formas da \emph{alt"-right} hoje. Mas, se nos
voltarmos para o fato de que o fascismo histórico, apesar da derrota
militar em 1945, seguiu permeando a política das democracias liberais no
pós"-guerra, podemos chegar a conclusões outras. Olhemos para os embates
do presente em sua expressão mais radical, sem moderações.

Qual a definição da \emph{antifa} contemporânea em relação ao que ela
combate sob o nome de fascismo? Segundo Bray, a partir das entrevistas
realizadas em sua pesquisa para a redação do livro, os \emph{antifas}
não se opõem ao fascismo por ele ser uma política basicamente
antiliberal, mas porque os grupo fascistas de hoje, alocados entre a
\emph{alt"-right}, promovem: a) o racismo e a supremacia branca; b) a
misoginia e o sexismo; c) o autoritarismo e discursos genocidas. Logo,
combatem práticas do presente largamente difundidas pela
\emph{alt"-right}. No Brasil, grupos anarco"-punks, como o A.C.R.
(Anarquistas Contra o Racismo), criado em 1993, promovem campanhas
regulares de alerta contra essas práticas -- especialmente contra grupos
de \emph{skinheads}, \emph{carecas} e \emph{white powers}. Entre os
anarco"-punks, destacam"-se as \emph{Jornadas Antifascistas}, que
acontecem regularmente todo mês de fevereiro, desde o ano 2000. A data
foi escolhida porque no dia 6 de fevereiro daquele ano, o instrutor de
cães, Edson Neris da Silva, foi espancado até morte por w\emph{hite
powers} na praça da República, em São Paulo, pelo simples fato de estar
de mãos dadas com seu namorado.

Episódios como este, que nos anos 1990 e 2000 eram tratados apenas como
de preconceito e brigas de gangues, evidenciam hoje, à luz do
crescimento da \emph{alt"-right} global, que os \emph{antifas}
anarco"-punks alertavam para um perigo real e para a urgência que tinham
em contê-los já em suas primeiras manifestações. Não era briga de
gangue, mas sim o fascismo tolerado pela democracia liberal. Ao tratar
as manifestações de racismo, homofobia e violência apenas como casos de
polícia, deixou"-se intocado o risco real e crescente alertado pelos
anarco"-punks às liberdades e a existência de pessoas diferentes do
modelo dominante nas sociedades contemporâneas. A mobilização em torno
do caso Edson Neris redundou na criação, em 2006, da Decradi (Delegacia
de Crimes Raciais e Delitos de Intolerância da Polícia Civil de São
Paulo). Especializada em ``crimes de intolerância'', sob a divisão de
homicídios, a delegacia não fez com que os grupos (neo)fascistas
deixassem de existir. Pelo contrário: desde então cresceram, chegando,
inclusive aos legislativos estaduais e federal em todo o Brasil. A
conclusão que se tira desse acontecimento menor e seus desdobramentos é
que não se combate o fascismo com a polícia e o direito penal, estes só
o reforçam em suas manifestações micro ou macro como política de
segurança. Os anarco"-punks, ao contrário, situavam a questão no campo
dos embate frontal, da ação direta.

Chegamos, com isso, à arena na qual se dá a luta entre fascismo e
antifascismo no século XXI: as políticas de segurança.

Não é necessário se estender muito para demonstrar a centralidade do
tema da segurança nas democracias no pós"-II Guerra Mundial, a ponto de
produzir no campo das Relações Internacionais a passagem dos Estudos
Estratégicos de guerra para o campo da Segurança Interacional. Com o fim
da Guerra Fria, vemos a expansão das políticas de segurança, tanto no
campo da segurança pública -- com programas urbanos como o tolerância
zero estadunidense, exportado para o mundo junto ao
super"-encarceramento, inclusive em governos de esquerda --, quanto no
campo da segurança internacional, pela ascensão das políticas de combate
ao terrorismo transterritorial, especialmente após o 11 de setembro de
2001. As políticas de guerra às drogas marcadas pela declaração de
guerra de Richard Nixon, no fatídico ano de 1973 expressão a junção
daquelas duas áreas. Também é amplamente conhecido o fato de que a
\emph{alt"-right,} em todo planeta, se viabilizou eleitoralmente com um
forte discurso sobre o setor de segurança: seja no norte, enfatizando o
combate ao terrorismo, a política anti"-migratória, a xenofobia e o
racismo; seja no sul, impulsionando a retórica do combate à
criminalidade, a degeneração moral da juventude, etc., além de sua
ligação direta com agentes de segurança no campo policial e militar, com
generais, capitães, cabos e soldados se tornando celebridades políticas.
Nesse sentido, a \emph{alt"-right} possuía um terreno preparado para a
sua expansão, especialmente após a crise global de 2008. A
\emph{alt"-right}, ao contrário do que muitos estudiosos do assunto
afirmam, não é antipolítica, mas a expressão da política contemporânea
levada ao seu paroxismo.

Desde a emergência das revoltas gregas em dezembro de 2008 até os
recentes atos antirracistas com presença de antifas nos EUA, um elemento
disparador dos motins é comum: a ação da polícia. Na Grécia, ainda que
os protestos estivessem direcionados contra as políticas de austeridade
do governo, a revolta antipolítica tomou conta das ruas após um
policial, Epaminondas Korkoneas, disparar contra um jovem anarquista de
ascendência armênia de 15 anos, Aléxandros Andréas Grigorópulos,
levando"-o à morte.

Não é demais lembrar que as jornadas de junho de 2013, ao menos em São
Paulo, foram atos que se iniciaram com os regulares protestos do MPL
(Movimento Passe Livre) contra o aumento da passagem e pela tarifa zero,
mas que teve o espalhamento da revolta e da indignação marcado pela
reação à violência da Polícia Militar no ato do dia 13 de junho de 2013,
quando manifestantes e integrantes da imprensa foram gravemente feridos.
Também não é fortuito que a proposta de abolição da polícia, antes
restrita ao campo dos abolicionistas penais em seus estudos
especializados ou proposições políticas, já circula entre os protestos
nas ruas dos EUA em cartazes, como os que levam a sigla A.C.A.B.
(``Todos os Policiais São Bastardados'', em português). E aqui demos a
volta completa, pois o uso dessa sigla provém precisamente da
contracultura de rua dos anarco"-punks, skinheads antifa e torcidas de
futebol antifa europeus, como alguns grupos hooligans e ultras.

O fascismo e o antifascismo do século XXI deve ser observado em torno
das políticas de segurança. Mais do que isso, o fascismo se realiza, em
sua metamorfose contemporânea, na centralidade que as democracias
concederam às variadas formas de securitização, judicialização com
regulação de direitos de minoria e penalizações a céu aberto, no
interior da governamentalidade planetária, que Edson Passetti nomeou de
\emph{ecopolítica} como segurança do vivo no planeta (Editora Hedra,
2019). E como, de fato, a história não se repete, não exatamente preciso
chamarmos esses regimes de fascistas. Em nosso laboratório no
Departamento de Relações Internacionais da UNIFESP, o LASInTec, chamamos
de ``democracias securitárias''. As expressões da extrema direita hoje
são antidemocráticas, quando se pensa no que deveria ser uma democracia.
Mas quando observamos a ascensão eleitoral desses grupos em países
democráticos, não seria um absurdo questionar se esta não seria a forma
contemporânea da democracia realmente existente, e se, ao invés de
defendê-la, não seria o caso de pensar em ultrapassá-la.

A revolta das pessoas comuns, anônimas entre praticantes da tática
\emph{black bloc} e grupos \emph{antifa}, contra as mortes perpetradas
por agentes de segurança é o fogo da antipolítica contra as políticas de
segurança, fogo que foi reanimado após a crise de 2008 entre gregos,
mesmo solo no qual o filósofo pré-socrático Heráclito, na antiguidade,
alertou: \emph{o fogo realiza}. Fogo!


\part{Nossa história: antifa e anarco-punk no Brasil}

\chapter{Um pouco sobre a história do \versal{ACR} -- Anarquistas Contra o Racismo}

\hfill{}\textsc{anarcopunkorg}

\bigskip

Criado no final de 1992, começo de 1993, o Projeto A.C.R. foi idealizado e hoje é tocado por anarquistas e em maior número por militantes do MAP (Movimento Anarco-Punk), embora esteja aberto a participação de tod@s.

O movimento anarco-punk já tinha a luta antifascista e anti-racista como uma das bandeiras de luta antes da idealização do A.C.R., cada localidade onde havia o MAP, já vinha desenvolvendo um trabalho dentro da luta antifascista e anti-racista, porém as dificuldades existentes na época eram este trabalho era desenvolvido de forma superficial, completamente desconectado dos movimentos sociais que também desenvolviam a luta anti-racista e sem contato com as “minorias” étnicas e de gênero atingidas pelo preconceito, discriminação e pelo racismo.

Mesmo entre os núcleos do MAP o trabalho anti-racista carecia de estruturação orgânica, coesão e de intercâmbio internúcleos. A partir desta análise, alguns indivíduos de São Paulo que não participavam necessariamente dos mesmos coletivos iniciaram uma articulação local que transformou-se no que hoje é o Projeto A.C.R.

Desde o início já tínhamos claro nossa opção por organizar um projeto, um trabalho em conjunto, uma idéia que tivesse fácil circulação entre os coletivos e entre os vários indivíduos. O primeiro panfleto surgiu quando ia ser realizado um pedágio para arrecadar fundos para realizarmos algumas atividades. Este panfleto refletia nosso repúdio ao racismo cotidiano e em especial as ações cometidas no ano anterior pelos “white-powers”, e foi assinado como “Anarquistas na Rua Contra o Racismo” (em março de 1993).

No mês seguinte (abril de 93), um adolescente negro de 15 anos foi morto pelo grupo nazi-fascista “carecas do ABC”, na região de São Bernardo do Campo na grande São Paulo.
Os movimentos sociais anti-racistas, que já haviam se reunido em outubro de 1992, em repúdio ao ataque dos “white powers” à Rádio Atual (de programação dirigida a comunidade nordestina em São Paulo) e em protesto ao programa “Documento Especial” que deu voz aos neonazistas de São Paulo, voltam a se reunir em abril de 1993 som um clima de indignação e revolta face ao assassinato deste adolescente negro. Esboça-se na cidade de São Paulo a criação de um amplo fórum multiétnico, supra-político e multi-cultural contra a ação dos neonazistas, nós enquanto anarco-punks e com o embrionários panfleto participamos das discussões.

Vários segmentos da sociedade estavam ali representados: movimentos negros, mulheres, nordestinos, organizações judaicas, grupos de pesquisa, movimentos populares e sociais entre outros.

A reunião deu-se no Conselho Participativo da Comunidade Negra de São Paulo e marcou para nós o início de nossos trabalhos a nível de A.C.R. e o lançamento da pedra fundamental para direcionarmos nosso trabalho para uma inserção social enquanto militantes anarquistas/anarco-punks na luta contra o racismo, preconceito e a discriminação em geral.
O resultado dessa reunião foi a realização de uma grande passeata com mais de 4 mil pessoas no dia 13 de maio de 1993. Panfletos e carros de som repudiavam a ação dos neonazistas, clamavam por justiça e alguns juravam vingança.
Passada a indignação inicial, os meios de comunicação deixaram de divulgar o caso, nenhum neonazista foi preso, a dimensão inicial do fórum anti-racista foi esvaziado e quase tudo voltava ao normal.

Nós dissemos quase tudo…

… Dias após a passeata do dia 13 de maio, um militante nosso foi atacado e violentamente espancado por neonazistas do grupo “carecas do ABC”, tomou vários pontos na testa, alguns dentes a menos e um pequeno afundamento facial.
Da nossa indignação face a mais esta agressão fascista, nasceu a convicção da necessidade de nos organizarmos enquanto anarquistas, anarco-punks e antifascistas e de nos articularmos com outros segmentos da sociedade, também alvos da violência neo-nazi-fascista.

Somente assim poderemos fazer frente a crescente ação de grupos de extrema direita e também desenvolvermos uma ação efetiva contra o racismo e outras formas de discriminação cotidianas.

Sabíamos que o Projeto A.C.R. e o MAP não poderia limitar seu combate a grupo skinheads, que apesar de violentos e homicidas em potencial, são apenas massa de manobra de setores ultra-reacionários, conservadores e nazi-fascistas da sociedade, nem entramos numa aspiral de violência com eles, pois estaríamos no campo deles, uma vez que a apologia e uso da violência, o culto a força física e a intolerância extremada são algumas das principais características dos skinheads.

Ao invés de nós entrarmos no campo da violência física com os fascistas, nosso projeto buscou e busca construir alianças com vários setores sociais para em conjunto traçarmos estratégias eficientes para coibir a ação da extrema direita e combater as manifestações cotidianas de racismo na nossa sociedade.

Todas estas constatações a partir de São Paulo encontraram terreno fértil em vários núcleos do MAP pelo Brasil, núcleos estes que posteriormente vieram a fazer parte do projeto A.C.R., no caso Santos, Rio de Janeiro e Curitiba.

A ampliação do Projeto A.C.R. a outras localidades foi um fato determinante para torná-lo mais coeso e produtivo, uma vez que o crescimento do nosso projeto sempre foi mais qualitativo do que quantitativo.

Os núcleos existentes atualmente Criciúma, Curitiba, Rio de Janeiro, Santos, São Paulo (e um núcleo colaborador em Cruzeiro – SP) buscam desenvolver trabalhos com as comunidades e movimentos sociais de suas regiões dentro de uma linha geral de conduta aprovada de forma horizontal por todos os núcleos.

Trabalhos estes que buscam resgatar a capacidade de auto-organização da sociedade despertando a população e os movimentos sociais anti-discriminatórios, para a necessidade de criarmos instâncias de luta contra todas as formas de preconceito, racismo e discriminação.

Defendemos com veemência a criação de fóruns multi-étnicos, supra-políticos-religiosos e multiculturais como forma dos vários segmentos atingidos pela intolerância racial da sociedade, reagirem e construírem instâncias sociais combativas e atuantes onde a democracia étnica seja um fato e não um engodo.

Buscamos com nosso trabalho, enquanto projetistas, um permanente aperfeiçoamento dos métodos de ação tornando mais visível, coeso e eficiente nossa resistência e nossa batalha.
Batalha esta que desenvolvemos de forma apaixonada, buscando superar assim uma a uma as adversidades que encontramos pelo caminho, e que não são poucas.
Batalha esta que travamos com nossos punhos cerrados, com nossa bandeira negra hasteada e com a disposição de quilombolas com anabolas que somos.

SOMOS O A.C.R.!!! SOMOS O M.A.P.!!! SOMOS OS ANARQUISTAS CONTRA O RACISMO!!

\hfill{}\emph{Setembro de 1999}


\chapter{O Movimento Anarcopunk e a luta anti-fascista no Brasil}

\hfill{}\textsc{Imprensa Marginal}

\bigskip

O Movimento Anarco Punk no Brasil é fruto de uma crescente politização dentro de parte da cena punk que se dá em meados dos anos 80 e início dos 90. Uma de suas principais bandeiras de luta, desde os primórdios, foi o combate ao nazi-fascismo, o racismo e o preconceito. Em muitas das localidades onde se formou o Movimento Anarco Punk, desenvolveu-se também um trabalho dentro da luta anti-fascista e anti-racista. Tal bandeira de luta já era levantada anteriormente no meio anarquista em diversas partes do mundo desde o surgimento das primeiras tentativas de ascensão de ideologias de extrema direita como o fascismo e o nazismo. No Brasil não foi diferente, e o surgimento do integralismo gerou um fervoroso combate por parte do movimento anarquista que a partir da década de 30 ganha muita intensidade.

O início dos anos 90 é marcado pela ocorrência de diversos casos de agressão e violência protagonizados por grupos de skinheads White Powers, Carecas do Subúrbio e Carecas do ABC. Esse contexto contava também com a aparição e evidência na mídia de políticos de extrema direita como Armando Zanini Junior, presidente do Partido Nacionalista Revolucionário Brasileiro (PNRB). Assim, desde o final dos anos 80 vão se intensificando cada vez mais as relações entre skinheads e organizações políticas integralistas e nazistas, culminando na entrada de Carecas nos quadros do PNRB e em ações públicas como, por exemplo, o evento de homenagem ao aniversário de cem anos do nascimento de Hitler, que ocorreu em 1989 com participação de Carecas do Subúrbio, Carecas do ABC, Ação Integralista e integrantes de outros partidos nacionalistas na Praça da Sé.

Em meio a este forte processo de tensão e buscando formas efetivas de combater a ação nazi-fascista destes grupos, anarcopunks partiram em busca de contatos com outros movimentos sociais e agrupações que também pudessem estar de alguma forma envolvidas no combate ao avanço da extrema direita. Surgia a percepção de que somente por meio desta parceria poderiam fazer frente à crescente ação dos grupos de extrema direita e desenvolver uma ação efetiva de combate. Em 1992, ante ao ataque de skinheads White Power à Rádio Atual, de programação dirigida à comunidade nordestina em São Paulo, e logo após a realização de uma edição do programa “Documento Especial” que deu voz aos neonazistas de São Paulo, diversos movimentos sociais anti-racistas se reúnem para uma discussão conjunta no mês de outubro. Entre novembro e dezembro o Movimento Anarcopunk realiza pedágios de rua para conseguir dinheiro para confecção de faixas e panfletos para uma campanha anti-fascista, e no dia 12 de dezembro é organizada uma passeata. Pouco depois, em 1993, com a morte do estudante negro Fábio dos Santos em Santo André, em decorrência de espancamento por 30 skinheads, este processo de atuação política do Movimento Anarco Punk se amplia, ocorrendo com maior força os contatos com outros movimentos sociais para parcerias de combate. Vai tomando corpo a criação de um fórum contra a ação dos neonazistas, com participação tanto de anarcopunks, quanto de diversos outros grupos, como movimentos negros, de mulheres, nordestinos, organizações judaicas, grupos de pesquisa, movimentos populares, entre outros. A reunião ocorreu no Conselho Participativo da Comunidade Negra de São Paulo, e o resultado foi uma grande passeata com cerca de 4 mil pessoas no dia 13 de maio de 1993. Grupos e movimentos punks, negros, feministas, e de atuação artística, cultural e política se reuniram em frente à embaixada sul-africana na Av. Paulista/MASP, para uma passeata anti-racista. A passeata passou pela Av. Brigadeiro e foi até a Praça da Sé, acabando com apresentação musical de diversas bandas.  No decorrer de todo o ato foram feitas muitas falas contra a atuação de grupos nazi-fascistas de Carecas e White Powers. Infelizmente, passado algum tempo, a dimensão inicial do fórum anti-racista foi esvaziado, e os meios de comunicação iam deixando de divulgar o caso.

Este momento marcou também o início dos trabalhos do projeto ACR – Anarquistas Contra o Racismo, que com o passar do tempo teve formação  de núcleos em diversas localidades para além de São Paulo – como Santos, Criciúma, Rio de Janeiro e Curitiba. A proposta era, dentro da cena punk, incitar a politização no que se refere à questão anti-fascista e, para muito além, estreitar laços com outros movimentos sociais e ampliar a rede de combate ao fascismo de forma concreta. Neste período há forte relação com movimentos LGBT, negros e judaicos, e realização de atividades diversas sobre a questão. Respeitavam-se as peculiaridades específicas de cada movimento, buscando construir a partir dos pontos de afinidade parcerias, compartilhamento de informações e apoio em ações de combate aos grupos e instituições nazi-fascistas nas diversas localidades.

Já era claro para o Movimento Anarco Punk e o Projeto ACR que o combate ao fascismo não poderia se limitar apenas a grupos skinheads, que em última instância eram apenas uma pequena parte de um problema muito maior, que envolvia setores diversos da sociedade com atuação em muitos âmbitos diferentes. O Projeto ACR também não acreditava no simples uso da violência como estratégia de combate, visto que a apologia e uso da violência, o culto à força física e a intolerância extremada são características próprias destes grupos de skinheads nazi-fascistas, e não seria possível combate-los a partir de práticas semelhantes. Ainda assim, as táticas de auto-defesa à esses grupos sempre estiveram em pauta, mas para além disso anarcopunks buscaram construir parcerias com vários setores da sociedade, para que em conjunto fossem traçadas estratégias eficientes para coibir a ação da extrema direita e combater as manifestações cotidianas de racismo na sociedade, propondo o respeito, a valorização da diversidade e da liberdade.

Diversos eventos públicos foram organizados nesta época, com mostras de vídeos, debates, palestras, panfletagens e ciclos de atividades anti-fascistas. Em novembro de 1994, por exemplo, foi organizado o Ciclo Anti-Fascista, uma série de três eventos com apresentações de bandas anarcopunks, palestras com a Unegro, Ben Abrahan (comunidade judaica ), passeata de rua e outras atividades. Foi um importante evento de discussão da luta anti-fascista que ia se tornando cada vez mais concreta e consistente.  Outro ciclo de atividades e debates sobre a questão foi organizado em 1995 em Curitiba pelo Grupo Anarquista Via Direta de Ação (GRAVIDA), contando com seis palestras, debates, exposições e outras atividades de 30 de outubro a 02 de dezembro. Também foram organizadas, no decorrer dos anos 90, diversas atividades de vídeo-debate anti-fascistas e eventos musicais contra o racismo.

Os núcleos ACR de cada localidade mantinham contato frequente entre si, realizando encontros gerais periódicos e fazendo circular os informes locais mensalmente, e organizaram diversas manifestações públicas e atividades de debate, editaram boletins e materiais de denúncia, produziram dossiês, e ainda criaram um forte canal de diálogo com a imprensa e outros movimentos. Agindo localmente, cada um dos grupos articulava materiais de denúncia e dossiês, realizava manifestações públicas e outras atividades; em conjunto, também organizavam campanhas coletivas, materiais impressos e outras ações que ultrapassavam as fronteiras de cada localidade. Dentre os casos que tiveram forte mobilização nos anos 90 estão a morte de Fábio dos Santos em Santo André em 1993; o assassinato de Carlos Adilson Siqueira por skinheads Carecas do Brasil em 1996 na cidade de Curitiba; a realização de um encontro neonazista de skinheads do Paraná também em 1996; dentre tantos outros. Em março de 96, alguns punks foram abordados pela polícia e, enquanto eram revistados, o moicano de um dos punks foi arrancado à faca por um policial que gritava “Oi!” e “Skin!”, dizendo “não gosto de punks e muito menos de negros”. Depois que os policiais saíram, alguns punks tiraram foto do camburão, prestaram queixa dos policiais e denunciaram na imprensa. O punk agredido levou vinte pontos na cabeça e dois dos policiais foram afastados conforme nota da imprensa oficial. O ACR desenvolveu campanha de denúncia sobre esta agressão.

A primeira edição da Parada Gay em São Paulo, em 1997, também teve participação ativa do Movimento Anarcopunk, que ficou diretamente envolvido na questão da segurança do evento no combate a possíveis ataques durante a manifestação.

Outra campanha que teve ampla e ativa participação dos núcleos ACR e anarcopunks da época foi a questão de Mumia Abu-Jamal, militante negro afro-americano que foi injustamente acusado pelo assassinato de um policial branco e, após um julgamento pautado em inúmeras inconsistências, permanece preso até os dias de hoje, completando mais de 30 anos no cárcere e a maior parte deste tempo no corredor da morte. Foram realizados eventos, debates, publicações e atividades diversas que pudessem dar visibilidade a este emblemático caso do racismo estatal.

Brasil afora, outros coletivos e iniciativas anti-fascistas foram se formando, como é o caso do Coletivo Monanoz, que surge em 1995 em Florianópolis por anarcopunks que tinham como intuito a formação de um grupo de estudos sexuais e realização de atividades de denúncia e combate à homofobia e ao nazi-fascismo. O frequente intercâmbio entre os grupos anti-fascistas gerou também campanhas conjuntas e parcerias. Em outubro de 1995, um encontro de grupos do Projeto ACR no Rio de Janeiro tem como resultado a união dos materiais de denúncia existentes em cada localidade para criação de um grande dossiê anti-fascista, a ser utilizado como instrumento de combate a ação dos grupos nazi-fascistas. Nessa época também se intensificam as discussões sobre a luta afro-punk e sua importância.

Em meio a esse trabalho ocorreram por diversas vezes casos de ameaças ou violência por parte de grupos neonazistas. Dois dos coletivos anarcopunks que sofreram ameaças foram o KRAP (Koletivo de Resistência Anarco Punk) e o Coletivo Altruísta, ambos tendo recebido cartas de ameaça assinadas por grupos skinheads em meados da década de 90. Houve também casos de agressão física, que tornaram cada vez mais importante a prática da auto-defesa por parte de militantes anarcopunks. Em geral, as ações coletivas de resposta também seguiram politicamente no sentido de tornar públicas as ameaças e agressões sofridas, denunciar o caráter nazi-fascista dos grupos de extrema-direita, e reafirmar o engajamento e comprometimento com a luta anti-fascista e anti-racista, o que gerou apoio direto de diversos grupos e indivíduos.

Por questões diversas, muitos dos núcleos do Projeto Anarquistas Contra o Racismo se dissolveram durante os últimos anos da década de 90, desta época restando ativo o núcleo de Criciúma/SC, que desenvolveu trabalhos ligados a esta questão em escolas, junto a comunidade LGBT, negra, pessoas usuárias de CAPS, entre outros, e possui um grande acervo de materiais anti-fascistas. As experiências que estes núcleos obtiveram no decorrer de seu trabalho, porém, foram grandes contribuições para as movimentações anti-fascistas que surgiriam a seguir. Ainda assim, as discussões e ações de denúncia e combate referentes à luta anti-fascista permaneceram vivas, seja por meio de fanzines e panfletos, discussões e debates, seja por meio de ações de rua e manifestações.

A década de 90 chegava ao fim e os anos subsequentes não seriam menos problemáticos. Diversos casos de agressão protagonizados por grupos de skinheads e nazi-fascistas se faziam frequentes, e um caso muito emblemático ocorre então em fevereiro de 2000: a morte do adestrador de cães Edson Neris, morto a chutes e golpes de soco inglês por dezenas de Carecas do ABC na Praça da República. O caso, de extrema brutalidade e intolerância, gera reações de repúdio e comoção de diversos grupos lgbt, de direitos humanos, agrupações punks e libertárias, ocorrendo manifestações conjuntas e atos diversos. Na ocasião, anarcopunks participam das mobilizações, organizando atividades de denúncia. A partir deste ano, começa a se formar a Jornada Anti-Fascista, que passa a ser organizada anualmente, durante o mês de fevereiro, e acontece até os dias de hoje. Inicia-se como uma manifestação de um único dia para, com o passar dos anos, tornar-se um mês inteiro de atividades sobre a questão, com apresentação de bandas, debates, palestras, vídeos, atos de rua e outros. Ultrapassando os limites de São Paulo, anarcopunks de outras localidades organizaram também atividades antifascistas durante o mês de fevereiro. A morte de Carlos Adilson em 1996, no mês de março, fará com que posteriormente, sejam organizados em Curitiba atividades do Março Anti-Fascista, com proposta semelhante à Jornada criada em São Paulo.

\chapter{``Aqui para ficar. Aqui para lutar'': O florescer de uma luta anti-fascista}

\hfill{}\textsc{marina knup}

\bigskip

\section{Parte I}

\subsection{Paki-bashing e racismo na Inglaterra dos anos 60-80}

Com o fim da Segunda Guerra Mundial, a Inglaterra passou a receber um grande número de imigrantes, com políticas de imigração que no início pautavam, sobretudo, a reconstrução da economia no pós-guerra. A partir de então vai se intensificando a entrada de pessoas das ex-colônias de regiões caribenhas e do sul asiático, como Jamaica, Índia, Paquistão e Bangladesh, e de regiões africanas como Kenya, Uganda e Nigéria.

As próximas décadas serão muito difíceis para estxs imigrantes, que vão se vendo imersxs em um ambiente de medo e violência racista. Em meio a um cotidiano de ataques violentos contra famílias inteiras de imigrantes, comércios, casas e vizinhanças onde viviam, muitxs tiveram suas vidas marcadas pelo racismo. Para estas famílias, o abuso e violência racial era parte da existência cotidiana na Inglaterra. E da indignação com esse quadro, floresceu uma intensa luta contra o racismo e o fascismo, protagonizada pelxs próprixs imigrantes e seus descendentes, que por meio da união e da auto-organização coletiva construíram sua forma mais efetiva de auto-defesa. Sem esperar pela ação do Estado ou da policia, a juventude asiática se empoderou por meio da ação direta e foi às ruas, tomando de volta as rédeas de suas vidas e construindo um movimento anti-racista inspirador.

Apesar de ser uma história muito difundida entre as comunidades asiáticas inglesas, com uma forte preocupação de resgate histórico e registro de memórias, a impressão que temos é que muitas vezes tudo isto acaba ficando restrito as próprias comunidades, estudiosxs do tema e alguns círculos ativistas. Este artigo é uma tentativa de resgate desta história em língua portuguesa, focando-se principalmente no período que vai do final dos anos 60 ao início dos 80. Em memória de todxs xs imigrantes perseguidxs pela intolerância fascista, e para que se mantenha viva a luta que levaram adiante por uma existência livre de racismo!

\subsection{Racismo x imigração}

A relação entre racismo e imigração na Inglaterra do século XX tem um histórico longo e complexo, ligada a fatores como a formação de sistemas de Estado-nação baseadas em livre mercado, nacionalismo e opressão colonial. O período pós Segunda Guerra Mundial é um momento em que a questão começa a se acirrar, pouco a pouco adquirindo uma dinâmica concreta na sociedade em um nível mais formal e institucional, por meio do estado, partidos políticos e organizações.

Já nos anos 50, a imigração torna-se uma pauta cada vez mais presente na agenda política, aos poucos surgindo o debate sobre a necessidade de controle da imigração negra/asiática na Inglaterra. Os argumentos anti-imigração sustentavam a vinda destas pessoas como uma ameaça ao modo de vida inglês, a lei e a ordem. Neste contexto, grupos de extrema-direita passam a ampliar sua propaganda anti-imigração, realizada abertamente. É a época também em que garotos brancos “Teddy Boys” perseguiam imigrantes negrxs, e verifica-se um crescimento de ataques violentos que culmina nas revoltas de Notting Hill em 1958, quando ocorreram inúmeros ataques a imigrantes caribenhxs e suas casas e uma revolta nas ruas.

Neste período surgem algumas iniciativas de diferentes âmbitos da sociedade para tratar das questões anti-racistas: em 58 é fundada a Indian Workers Association no Reino Unido, que fazia campanhas contra a discriminação e tinha também boas relações com o movimento inglês de sindicatos; no mesmo ano o Institute of Race Relations inicia seus trabalhos; e em 64 surge também a Campaign Against Racial Discrimination, com o objetivo de reivindicar legislações anti-discriminatórias junto ao governo trabalhista.

Em meados dos anos 60 a problemática vai se intensificando. Neste contexto, uma das maiores personificações públicas deste discurso conservador foi o parlamentar Enoch Powell, que realizou uma série de discursos entre 1967 e 1968, focados em denunciar “os perigos” da imigração, atraindo enorme atenção midiática e forçando reações do governo trabalhista. Seu discurso mais conhecido e apoiado, de abril de 1968, foi intitulado de “Rios de Sangue” [Rivers of Blood]. Powell argumentava que a imigração levaria a “uma transformação total, desconhecida em centenas de anos na história da Inglaterra”, e que o país seria banhado por “rios de sangue” caso a vinda de imigrantes negrxs e asiáticxs não fosse detida. Este discurso, feito poucos dias antes do Parlamento começar a deliberar a Lei de Relações Raciais, teve ampla divulgação midiática, gerando pânico e medo. Powell atacou a lei que regulava a igualdade de tratamento com relação à moradia e emprego, argumentando que imigrantes ficariam em uma posição privilegiada, falando em uma invasão que tomaria áreas inteiras.

Para além das demandas por restrições, Powell e seus correligionários reivindicavam a repatriação, argumentando que o número de imigrantes de ex-colônias era muito grande e, neste caso, deveriam ser mandados de volta. Os discursos de Powell e outros e a enorme atenção midiática que tiveram, acabaram por mudar os termos do debate político quanto a pessoas negras/asiáticas e relações raciais, destruindo o frágil “consenso bipartidário” sobre raça e imigração criado pelo governo trabalhista. Neste contexto, uma série de legislações que controlavam a imigração foram surgindo, como o Immigration Act de 1971, por exemplo.

\begin{quote}
Existe um inimigo muito maior. Este inimigo é o Estado Inglês, e as armas que estão usando contra nós são as Leis de Imigração. Centenas de pessoas estão sendo jogadas nas prisões e deportadas. A polícia está fazendo batidas diárias em nossas casas e a situação irá ficar pior com as novas propostas dos Tory.(…) As leis de imigração são racistas. Eles as tem introduzido não apenas para manter os que já estão aqui como forças de trabalho colonial dóceis e submissas. Os Oficiais de Imigração e a policia são os novos administradores coloniais como os ingleses eram em nossos países. Estas leis nos degradam e abusam. São usadas para nos molestar e explorar. Elas tiram nossa dignidade humana e nos tratam como animais (…)

Kala Tara \#1, publicação AYM Bradford, 1979.
\end{quote}

Logo após o discurso dos “Rios de Sangue”, a quantidade de ataques violentos contra imigrantes cresceu dramaticamente por toda Inglaterra. No entanto, era comum que se colocasse essa violência como um resultado da presença dessas comunidades, representadas como fontes potenciais de violência e não como suas vítimas. Estes argumentos legitimavam a violência racista, sustentando que eram xs imigrantes que atacavam as pessoas brancas inglesas supostamente tomando suas áreas e tirando-as de suas casas. Frases como “invasão sem precedentes” e criação de “áreas alienígenas” eram muito comuns. O “pânico moral” gerado pela mídia e políticos anti-imigração com relação à presença de imigrantes contribuiu imensamente para o surgimento de novas forças políticas explicitamente racistas, que pouco a pouco colocavam imigrantes como bodes expiatórios para problemas econômicos e sociais que o país sofria no momento.

O imigrante bengalês Suroth Ahmed relembra:

\begin{quote}
No começo quando cheguei à Inglaterra, Enoch Powell era visto como uma pessoa má. (…) Nós o víamos falando na TV. Nós asiáticxs e povo Bengali o odiávamos muito. Enoch Powell era uma pessoa ruim, assim como os skinheads que atacavam (durante o governo de Heath\footnote{Edward Heath, foi primeiro ministro da Inglaterra durante os anos de 1970 a 1974.}), a comunidade negra e asiática. Considerávamos ambos como ruins. No início não podia compreender seu discurso, mas eu estava convencido, ele não era um bom homem. Estava sempre nas manchetes da televisão. Eu não sabia inglês o suficiente, mas os anciãos nos diziam que tentava deportar o povo negro e asiático.
\end{quote}

\subsection{Paki-bashing?}

\epigraph{A Grã-Bretanha tomou conhecimento do termo paki-bashing pela primeira vez na última quarta-feira. Um grupo de skinheads se gabava pela TV de ter espancado imigrantes na Zona Leste de Londres, por pura diversão}{\textit{Jornal Sunday Mirror}, 1969}

“Paki-Bashing” era o termo utilizado para descrever a perseguição violenta a imigrantes asiáticxs na Inglaterra. Embora na mentalidade racista estas pessoas fossem vistas como um grupo étnico uniforme, na realidade vinham de diversos contextos e localidades: havia Sikhs da região Punjab e hindus da região Gujarat da Índia, muçulmanas do Paquistão, Bengalis, entre outras. “Paki” é um termo considerado altamente racista, usado para se referir tanto a paquistaneses quanto a todxs xs outrxs imigrantes sul asiáticxs de diferentes origens, que eram tidxs como “a mesma coisa”.

\begin{quote}
Eu cheguei em 1968, e logo que cheguei neste país fui confrontado pelo paki-bashing, fui confrontado pelo movimento skinhead e todas estas pessoas. (…) Embora paki fosse para paquistaneses, todos os indianos, bangladeshis, todos para eles eram a mesma coisa. Eles costumavam odiar a gente, queriam nos levar para fora desta área. (…) Eles costumavam cuspir na gente e nos batiam nas ruas. E naquela época nosso povo não era tão forte. Porque tinha apenas vindo de seu país de origem, e os asiáticos sempre foram pessoas muito pacíficas. E nosso povo veio para cá para ganhar algum dinheiro, para apoiar suas famílias em casa. Não vieram para cá para lutar com ninguém. Então a resistência não era tão boa, não era tão forte.

Mahmud Rauf, imigrante bangladeshi
\end{quote}

O racismo que atingiu sistematicamente estes grupos étnicos na Inglaterra se deu de forma extremamente violenta. Registros diversos apontam que se dava por meio de linchamentos, assassinatos, bombas, ataques, perseguições, intimidações e expulsões. Os ataques atingiam famílias de imigrantes em suas casas e locais de trabalho, bombas eram colocadas em caixas de correio e pedras quebravam as janelas; era comum que jovens asiáticos fossem esfaqueados ou espancados; ovos e tomates eram jogados contra mulheres e crianças; pequenas garotas eram agredidas na rua por jovens brancos que saíam correndo aos risos; muitas pessoas, com medo de sair, se tornaram prisioneiras de suas próprias casas. Embora espalhados em variados registros, são muitos os relatos de imigrantes e suas famílias sobre este cotidiano, retratados em entrevistas, documentos de denúncia, periódicos, filmes de ficção, livros, documentários e letras de música. Em todos eles, questões como racismo, xenofobia e “paki-bashing” se repetem, demonstrando o quanto estes aspectos estavam presentes em suas vidas.

\begin{quote}
(…) Era muito assustador, e muitas das pessoas que apenas tinham vindo de Bangladesh não queriam se envolver em brigas e nada do tipo. Elas costumavam ficar confinadas em suas casas. As únicas ocasiões em que saíam era porque precisavam ir ao trabalho, ir fazer compras… Fora isso não queriam sair. As pessoas de nosso país, sempre quando chegavam viviam em grandes alojamentos. E esses lugares eram sempre cheios de histórias de espancamentos de pessoas bangladeshi por pessoas brancas, por skinheads, por… Quando digo pessoas brancas, não quero dizer as pessoas boas, mas as pessoas brancas más, da BNP, e… Sempre havia espancamentos, havia histórias de idosos espancados, com seus dentes quebrados… este tipo de coisa era muito comum naquela época. (…) Eu nunca tive nenhuma experiência de espancamento, mas pelo menos um par de vezes cuspiram em mim.

Mahmud Rauf, imigrante bangladeshi
\end{quote}

Embora o termo “wog-bashing” tenha sido usado para descrever violências deste tipo protagonizadas por Teddy Boys nos anos 50, a palavra “paki-bashing” é citada por muitxs autorxs como tendo surgido entre o final da década de 60 e início de 1970. Segundo Benjamin Bowling, a violência em si pode ser associada a três fatos inter-relacionados: “o amplo pânico moral quanto à imigração e raça estimulado por Powell e seus correligionários, a fundação em 1967 e posterior apoio conferido à National Front, e o surgimento de uma nova, violenta e explicitamente racista cultura juvenil: os skinheads”.

A primeira grande onda de “paki-bashing” ocorreu a partir de 1969/1970. A popularização do termo na imprensa ocorre neste período. Alguns estudos apontam o assassinato de Tosir Ali em abril de 1970 como momento em que se passa a veicular o termo na mídia, mas existem também alguns registros de jornais que em 1969 já faziam seu uso. Segundo a pesquisadora Anne Kershen, o termo já havia sido usado antes um pouco além da fronteira de Spitalfields, no complexo habitacional Collingwood em Bethnal Green, em 1968. Segundo ela, “Isto corresponde à época em que agressivos jovens brancos, de cabeças raspadas, ficaram conhecidos como ‘skinheads’”.

Durante esta primeira onda de paki-bashing que assolou a Inglaterra, o caráter dos ataques racistas protagonizados por skinheads se deu muito mais a partir de expressões violentas de xenofobia, nacionalismo extremado e opressão colonial, com pouquíssima relação com partidos políticos de extrema direita como a National Front, que só ganharia visibilidade anos depois. Bowling sustenta que os skinheads tinham um forte senso de território que estava intimamente ligado ao paki-bashing como um “ritual e defesa agressiva da homogeneidade cultural da comunidade contra os mais óbvios bodes expiatórios estrangeiros. Neste sentido, é comum encontrar justificativas aos ataques no fato dxs imigrantes asiáticxs possuírem costumes culturais, alimentares e sociais muito diferentes dos ingleses, não falarem inglês corretamente, assim como argumentos ligados a disputa de empregos, moradia e oportunidades com inglesxs brancxs. Em muitos casos, estes jovens declaravam ver o paki-bashing como mera diversão – uma “diversão” sistemática que gerava mortes, ferimentos físicos e pânico. Ironicamente, enquanto tinham nxs asiáticxs seu foco mais direto de violência, ouviam e dançavam música caribenha/jamaicana rock-steady, reggae e ska.

Já nessa época, imigrantes começam a pensar em formas de auto-defesa, visto que a resposta da polícia era praticamente nula. Um registro de 1968 relata que ao voltar para casa após o trabalho de bicicleta, o imigrante Gulam Haider Ellam avistou uma gangue de skinheads esperando por ele. Pedalou rápido para tentar escapar, e depois disso passou a andar com uma barra de ferro para se defender, enquanto motoristas de ônibus asiáticos andavam também com tacos de hockey durante as viagens. Em meados dos anos 70 estas formas de auto-defesa ficarão mais organizadas, tornando-se pautas coletivas.

1970 será um ano marcado por centenas de ataques racistas, que tornaram o “paki-bashing” notícia nacional. Neste período, diversos casos de agressões com garrafadas, chutes, tijoladas, facadas, espancamentos, e outras formas de ataque foram divulgados. Ocorreram invasões em regiões de comunidades asiáticas como Brick Lane e Southall por gangues skinheads que tumultuavam e feriam moradorxs, por vezes com presença de centenas de agressores. De março a maio de 1970, 150 pessoas foram atacadas somente na região de East End de Londres, porém aconteceram casos semelhantes em Wolverhampton, Luton, Birmingham, Coventry, West Bromwich, entre outros. Na mesma época, o Sunday Times dedicou uma página inteira à perseguição racista que atingia milhares de paquistanesxs na região do Spitalfields Market, revelando que em janeiro e fevereiro daquele ano houve ataques regulares de skinheads contra paquistanesxs em Spitalfields. Já o Observer, advertia que “Qualquer asiático descuidado o suficiente para andar sozinho pelas ruas a noite era um tolo”. A questão chegou a atrair inclusive atenção internacional: um canal de televisão francês, por exemplo, fez um pequeno documentário em 1970 em Londres sobre as agressões à comunidade asiática, entrevistando imigrantes, skinheads e políticos.

Em 3 de abril de 1970, Tosir Ali foi assassinado à facadas em Bow, leste de Londres, a poucos metros de sua casa. Sua morte teve enorme visibilidade pública, e neste contexto o sindicato Pakistani Workers Union reivindicou um inquérito sobre o fracasso da ação policial. Uma declaração da Policia Metropolitana, entretanto, negou que pessoas negras fossem atacadas com mais frequência do que as brancas. Esta morte demonstra ainda a tendência da polícia em responsabilizar a própria comunidade asiática pela violência que sofriam: no dia seguinte após sua morte, um paquistanês que reclamou na televisão pela falta de ação da polícia pelos ataques foi preso e questionado por muitas horas sobre a morte de Ali.

Diversas manifestações, e até mesmo uma greve, foram realizadas por imigrantes asiáticxs contra a violência que sofriam na virada da década de 1960 e início dos 70. O New York Times de 25 de Maio de 1970, por exemplo, noticiou que no dia anterior “muitas centenas marcharam até a residência do Primeiro Ministro Wilson para pedir por proteção aos ataques de skinheads”.

\begin{quote}
A primeira geração a frequentar as escolas na Inglaterra passou por abuso racial, ataques racistas e discriminação como uma característica cotidiana da vida, tanto nos playgrounds e a caminho ou na volta da escola, quanto nas ruas. Enquanto seus pais lutaram e trabalharam para sustentar suas famílias, crianças asiáticas e jovens estavam aprendendo a viver e lidar com um ambiente diferente sem nenhum apoio familiar ou comunitário contra este racismo evidente.\footnote{Balraj Purewal, membro fundador da Southall Youth Movement in THE ASIAN HEALTH AGENCY – Young Rebels: The Story of the Southall Youth Movement.}
\end{quote}

A experiência de frequentar a escola poderia ser muito difícil para umx jovem descendente de imigrantes asiáticxs naquela época. Dentro da escola, a caminho de casa ou nos playgrounds onde brincavam, paki-bashing era algo muito real. Em 1972, um garoto Sikh foi esfaqueado em uma destas escolas por garotos brancos que disseram que o matariam caso não cortasse o cabelo e parasse de usar turbante. Outro garoto de 11 anos, Sohail Yusaf, foi espancado a caminho de casa depois da escola, ficando inconsciente em um canteiro de obras.

Segundo Saeed Hussain, “Na maioria das tardes de sexta, mas particularmente no fim do período escolar, feriado, ou o último dia antes do feriado eram dias comuns de paki bashing”. Assim, a permanência destas crianças e jovens na escola em alguns casos podia ser muito curta, como relata o imigrante bengali Mohammed Abdus Salam:

\begin{quote}
Eu vim de Bangladesh, então Paquistão Oriental, em 1969 com meus pais. Me matriculei em uma escola secundária local, a Montefiore Secondary School. Frequentei a escola por apenas alguns meses, não pude continuar devido aos ataques racistas, abuso racista e ataques de rua por skinheads e colegas brancos.
\end{quote}

As que permaneciam, encontravam sua sobrevivência na união com outrxs garotxs asiáticxs, uma união que a médio prazo teria bons frutos para a comunidade. Tariq Mehmood relembra: “Éramos muito conscientes do fato de que tínhamos de estar juntxs porque não podíamos pegar os ônibus (…) éramos atacadxs quando pegávamos os ônibus, e éramos atacadxs quando saíamos dos ônibus. E a única forma que tínhamos para sobreviver era conhecendo muitxs amigxs de outras escolas. Talvez por isso tantxs de nós ainda tenhamos contato até hoje.”

Em agosto de 1972, milhares de asiáticxs foram expulsos de Uganda, cerca de 50 mil delxs tendo passaportes ingleses – o que gerou enorme pânico na imprensa e colocou alguns políticos como Powell no centro do debate. Cerca de 28 mil asiáticxs de Uganda entraram no país, com grande comoção de organizações anti-imigração como o Partido Conservador e organizações como o British Movement e a National Front. Em uma matéria escrita por Simon Wooley sobre este momento, ele relembra: “A cidade em que cresci, Leicester, recebeu muitxs refugiadxs de Uganda, com ressentimento e reação que teria efeitos profundos em muitas pessoas. “Paki-bashing”, como era conhecido então, era levado a cabo por skinheads com o único propósito de aterrorizar comunidades asiáticas. Em resposta, xs asiáticos formaram a Sapno Gang – a gangue dos sonhos – cuja razão de existência era defender suas comunidades.”

Husna Matin, uma imigrante que se mudou com toda a sua família para a Inglaterra em 1974, se lembra que: “Os skinheads nos deram tempos muito difíceis. Bengaleses tinham medo de sair tarde da noite. Eles arrumavam menos problemas com mulheres, mas estavam atacando os homens bengali. Os skinheads costumavam bater nos bengali e roubar o que tivessem sempre que podiam. (…) Não tínhamos telefone na região. Algumas pessoas tinham de ir a New Road para fazer ligações e não se sabia se você poderia voltar seguro para casa. Os skinheads atacaram muitos bengaleses em seu caminho de volta das cabines telefônicas.”

Há um novo pico de ataques racistas violentos a partir da cobertura midiática sobre a chegada de imigrantes asiáticos de Malawi em maio de 1976. Em um contexto de declínio econômico e grandes taxas de desemprego, estrangeirxs são mais uma vez os bodes expiatórios para os problemas do país. Lojas, casas e organizações comunitárias foram apedrejadas e incendiadas no leste de Londres e Southall, mulheres de Newham tiveram seus saris (vestes femininas) queimados, ocorreram ataques racistas com cachorros, uma série de esfaqueamentos e foram registrados assassinatos em diferentes regiões.

Com o crescimento de organizações como a National Front neste período, para além de uma intensa atividade de propaganda racista, passam a ocorrer muitos casos de violência protagonizados diretamente por membros de partidos políticos de extrema direita. Um relatório apresentado em meados dos anos 70 por um membro do parlamento, Paul Rose, denunciava mais de 1000 incidentes de violência com envolvimento de grupos de extrema-direita como a National Front. Outras evidências relacionam a NF a diversos ataques a bomba, panfletagens racistas e ataques ao Community Relations Office em Londres. A NF organizou comícios de rua com propaganda abertamente racista, distribuíram panfletos racistas em regiões caracterizadas pela grande quantidade de moradorxs imigrantes, como Brick Lane, entre outras ações públicas. Há também relatos que apontam o racismo institucional em escolas, locais de trabalho e diversas outras situações e ambientes do cotidiano das comunidades imigrantes.

Mas estxs jovens que chegaram à Inglaterra pequenxs estavam crescendo, e esta segunda geração agiria de forma muito mais combativa contra todo este contexto.

\begin{quote}
Diferente da resposta muda dos mais velhos à discriminação racial e ataques, e suas aspirações de “voltar para cada um dia”, xs jovens asiáticos que vieram para estudar se viam como iguais a seus colegas brancos, vendo a Inglaterra como seu lar e que “estavam aqui para ficar”. Não estavam preparadxs para aceitar ou “virar a cara” para os constantes ataques e discriminação, e reivindicaram direitos iguais e justiça.

Jovens asiáticxs começaram a organizar e criar redes e alianças entre diferentes escolas e vizinhanças para defender e proteger a si mesmxs contra ataques racistas e apoiarem-se nos problemas sociais e culturais comuns que afetavam as comunidades. Estas redes e alianças eram a infraestrutura embrionária que levaria xs jovens a consolidar e formalizar o Southall Youth Movement (SYM).\footnote{Balraj Purewal, membro fundador da Southall Youth Movement in THE ASIAN HEALTH AGENCY – Young Rebels: The Story of the Southall Youth Movement.}
\end{quote}



\section{Parte II}

\subsection{Auto-organização e auto-defesa antirracista}

\epigraph{(…) para muitas da segunda geração de pessoas asiáticas que cresceram na Inglaterra durante os anos 70 e 80, a religião não foi um aspecto primário de definição de suas identidades. O assunto chave foi o racismo, que confrontavam na escola, nas ruas, nos lugares onde suas famílias poderiam morar e trabalho, produzindo uma ampla identidade anti-racista em torno do qual se organizaram}{\textit{Secular Identities and the Asian Youth Movements}, Dr Anandi Ramamurthy, University of Central Lancashire}

Entre 1973 e 1977, vão surgindo algumas tentativas de formar redes de comitês anti-fascistas/anti-racistas que envolviam pessoas e grupos dos mais diversos contextos. Em junho de 74, anti-fascistas organizaram uma passeata até o centro de Londres, contrapondo um comício da National Front. A manifestação acabou em conflitos com a polícia e membros da NF, e o manifestante Kevin Gately foi assassinado. Em fevereiro de 75 é criada a revista antifascista Searchlight, com o objetivo de contestar o apoio eleitoral à NF, seguida pelo surgimento do jornal Campaign Against Racism and Fascism (CARF), publicado pela primeira vez no final de 76 pelo Richmond \& Twickenham Anti-Racist Committee e a partir de 77 adotado como jornal da All London Anti-Racist, Anti-Fascist Co-ordinating Committee, uma federação formada por 23 comitês anti-fascistas locais. Em 77 também surge a Anti-Nazi League, que com o passar dos anos tem um grande crescimento, com bases em diversos setores sociais, entre outras organizações. Nesse momento também cresce a oposição à NF por parte do Labour Party, que avaliando o crescimento do sucesso eleitoral da Front concorda com uma campanha contra o racismo e reivindica a revogação das leis de imigração de 1968 e 1971.

Em junho de 76, a morte de Gurdip Singh Chaggar, de 17 anos, teve grande repercussão. Ele foi morto nas proximidades do Dominion Theatre em Southall, um local que era visto como símbolo de auto-organização pelxs sul asiáticxs que viviam ali, num claro ataque à comunidade como um todo. O Comissário de Polícia Robert Mark declarou na época que o motivo não era necessariamente racista, e como resposta foi organizado um comício público pela Indian Workers Association no domingo seguinte.

Kuldeep Mann relembra: “Me lembro do choque na comunidade. Sim, foi uma experiência pessoal muito viva para mim. Nós fomos ao Dominion Centre, que era um grande cinema naquela época, e seu corpo foi colocado lá fora e fomos todxs olhar. (…) a comunidade estava muito unida em sua tristeza, e as pessoas se sentiam com muita raiva. As pessoas mais jovens particularmente. Sim, foi um ponto de virada em minha memória, e para muitas pessoas em Southall também.”

Jovens marcharam até a delegacia de polícia pedindo por proteção das violências racistas, e declararam: “Lutaremos como leões”. Cercaram a delegacia, e se recusaram a sair até que dois homens asiáticos que haviam sido presos durante as manifestações fossem soltos. A demanda foi atendida, e outro encontro foi organizado logo depois para organizar unidades de auto-defesa. É neste contexto que nasce a Southall Youth Movement, um movimento de autodefesa que surgia como resposta a segunda grande onda de paki-bashing que assolava os bairros imigrantes ingleses.

A inquietação causada em Southall após esta morte em 1976 foi um grande estopim para o desenvolvimento de diversos movimentos de luta anti-racista em muitas comunidades asiáticas, que percebiam que somente por meio da auto-organização poderiam se defender. Se para a geração anterior de imigrantes, havia sempre a incerteza quanto à permanência na Inglaterra e um sonho de retorno à terra natal, para essa nova geração de jovens, isto era diferente. Sentiam que era necessário lutar para que pudessem viver em paz neste lugar que também era delxs. Cada vez mais aumentava a desconfiança nas autoridades brancas, e a partir daí, a necessidade da auto-defesa nas comunidades. A juventude asiática passava a buscar na ação direta uma forma de proteção contra ataques racistas e discriminação por parte das autoridades, sem esperar mais que o Estado fizesse algo, e desafiando também as concepções da geração anterior. Era hora da juventude se organizar, e milhares de jovens asiáticxs se aproximaram desta luta.

Embora as pessoas envolvidas em geral nos AYM fossem asiáticas, havia uma identificação enquanto pessoas negras em uma sociedade branca, que lhes dava um sentimento de união com imigrantes africanxs e caribenhxs que passavam pelas mesmas experiências de racismo. Nesse contexto, o termo negro possibilitou uma identidade coletiva e solidariedade para levar adiante tanto a luta contra o racismo nas ruas, quanto contra suas expressões institucionais e políticas.


A proposta de se introduzirem enquanto juventude asiática, embora Southall fosse predominantemente Sikh e Indiana, tinha como objetivo a inclusão de toda juventude sul asiática, africana e caribenha. Sobre isso, Balraj Purewal comenta: “nos chamávamos de Southall Youth Movement, porque não éramos uma minoria em Southall. Falar em asiáticos poderia soar como se fosse algo especial, mas éramos a juventude de Southall.”

Outra questão importante era a percepção que tinham de relação entre racismo e opressão de classe. Como disse Tariq Mehmood, um membro da AYM Bradford, “para nós a questão não era só a pele escura, estava ligada a questões de classe. A maioria de nós era composta por trabalhadorxs e filhxs de trabalhadorxs. Para nós, raça e classe eram inseparáveis”.

A partir desta experiência do SYM, diversas iniciativas semelhantes começam a surgir em outras comunidades. Ainda em 76 surge a Asian Youth Organisation (AYO) em Bolton e Blackburn, e o embrião da Bangladesh Youth Movement (BYM); em 77 nasce a Asian Youth Movement (AYM) em Bradford, em um primeiro momento com o nome de Indian Progressive Youth Association (IPYA); em 78 a Asian Action Group de Haringey; em 79 a Asian Youth Movement de Leicester e Manchester; e o começo dos anos 80 vai ampliando o movimento em outras localidades: surge a Asian Youth de Birmingham e a United Black Youth League de Bradford em 81, e em 82 a AYM Sheffield. Aos poucos os AYMs e outros movimentos semelhantes iriam se espalhando por East London, Luton, Nottingham, Leicester, Manchester, Sheffield, Burnley, Birmingham, Pendle, Watford e outros lugares.

Segundo Tariq Mehmood, o que queriam “conseguir com a formação da AYM era realmente simples, queríamos ser capazes de nos defender, queríamos ser capazes de unir nossas famílias que para muitxs de nós tinham sido divididas por leis de imigração.” Rajonuddin Jalal, da BYM, diz ainda: “era uma organização que mobilizou as pessoas jovens, mobilizou a comunidade como um todo, deu voz à comunidade, organizou as pessoas para que apoiassem aquelas que eram vítimas de ataques racistas em suas casas ou nas ruas. Depois também teve o papel de politizar a comunidade.”

Em 77, a Southall Youth Movement ocupou um prédio na região, criando no local um centro da juventude que realizava diversas atividades, inclusive esportivas, e também dava ajuda legal e representação para as pessoas que tivessem problemas com a polícia. No decorrer do tempo, a SYM também mobilizou xs membrxs para ajudar outras comunidades, como os bangladeshis de Brick Lane no East End de Londres, a se defender dos ataques semanais. Este apoio da defesa física da comunidade também inspirou a formação de outros grupos locais.

No decorrer dos anos 70, aponta-se o crescimento da consciência de diversos setores da sociedade inglesa quanto à existência da violência racista, o que gera certa pressão para que entrasse em debate nas agendas públicas. Desta forma as denúncias de incidentes racistas se ampliam, com existência de cobertura na imprensa, e reivindicações pela ação do Estado chegam ao Parlamento e mesmo à residência do Primeiro Ministro em Downing Street. Entretanto, a criminalização de comunidades imigrantes e a introdução de uma nova lei de imigração serão assuntos que estarão no foco no jogo político deste momento. O final dos anos 70 será marcado também pela entrada de Margaret Thatcher como Primeira Ministra, com posicionamentos anti-imigração e políticas liberais que nada ajudaram no cotidiano violento de imigrantes asiáticxs na Inglaterra.

Porém também ganharão visibilidade diversas organizações anti-racistas se opondo à marchas da National Front por todo o país e denunciando a violência racista. Para barrar a NF nas eleições suplementares de 77 e 78, anti-fascistas espalharam milhares de panfletos que alertavam a ameaça fascista, contra-marchas foram organizadas, e muitas mobilizações aconteceram, conseguindo derrotar os candidatos do partido. Manifestações enormes com mais de 20 mil pessoas também foram organizadas no final da década denunciando o caráter racista das leis de imigração, com a formação da Campaign Against Racist Laws, que integrava movimentos da juventude asiática, caribenha, e organizações anti-racistas diversas. Nestas manifestações houve grande presença de mulheres asiáticas, que também denunciavam o sexismo destas leis.

\subsection{O racismo policial e criminalização das comunidades}

\begin{quote}
A nova geração do povo bangladeshi começou a crescer, e estas pessoas começaram a fazer cursos de auto-defesa, cursos de kung-fu e todas estas coisas. (…) Naquela época quando as pessoas estavam desamparadas, não queriam resistir, elas não tinham ajuda suficiente da policia local, a policia não as apoiava. E devido a toda essa não cooperação da lei e da ordem, nossxs garotxs jovens, a geração mais jovem, começou a resistir, construir sua resistência, como auto-defesa e outras coisas. E havia alguns incidentes acontecendo aqui e ali, estavam resistindo, revidando. Quando se era atacadx, elxs começaram a atacar! E então as autoridades pensaram “meu deus! Elxs não vão mais aceitar isto, estão resistindo! E quando começaram a resistir, eles pensaram “talvez esteja surgindo um problema”. (…) Eu lembro de um amigo, ele era especialista em kung fu, faixa preta, vermelha, não me lembro. Ele começou a mobilizar as pessoas desta área, que viviam nos arredores de Brick Lane. E então a policia começou a prestar atenção, “essxs garotxs estão começando a resistir, é melhor fazer algo!

Mahmud Rauf, imigrante bangladeshi
\end{quote}

Em 1978, o Bethnal Green and Stepney Trades Council publicou um importante documento sobre a violência racial em East End, baseado em registros da Bangladesh Welfare Association, Bangladesh Youth Movement e Tower Hamlets Law Centre. O material analisa mais de 110 ataques contra asiáticxs e suas propriedades que ocorreram entre janeiro de 1976 e agosto de 1978, tornando-se um dos documentos mais aprofundados sobre a questão. Entre os casos analisados estão esfaqueamentos, rostos cortados, pulmões perfurados, ferimentos de bala, espancamentos com tijolos, paus, guarda-chuvas, chutes, e outros métodos que quase sempre levavam as vítimas ao hospital.

Uma das questões colocadas no relatório, já anteriormente denunciadas em outros materiais, é a conivência policial e criminalização dxs imigrantes. Há inúmeros casos em que a polícia tentou evitar registrar as queixas de ataques racistas, negou a natureza racista da situação, não tomou nenhuma atitude ou, ainda, criminalizou as próprias vítimas.

Um artigo relata que em meados dos anos 70 em Bradford, os rumores de que mulheres paquistanesas foram atacadas por homens brancos eram impossíveis de ser confirmados, porque a polícia não queria dar informação de casos de mulheres asiáticas atacadas por homens brancos, argumentando que isso iria inflamar a comunidade asiática. Em contraposição, a policia não hesitava em dar informações sobre ataques contra mulheres brancas se o agressor fosse negro.


Como relembra a imigrante Clare Murphy, “(…) onde havia muitos ataques em Brick Lane contra Bangladeshis por skinheads ou outros racistas, o que geralmente acontecia era que o ataque era feito, alguém ligava para a polícia, provavelmente um Bangladeshi. A policia levava muito tempo para chegar lá, e quando chegava os agressores já tinham fugido, provavelmente ouvindo as sirenes. Bangladeshis da vizinhança saíam, possivelmente com pedaços de madeira ou algo para se proteger. E eram elxs que eram presxs por porte de armas. (…) Não havia proteção policial suficiente; diziam que iam prender Bangladeshis com pedaços de madeira ou o que fosse. A polícia fez insinuações sobre xs Bengalis serem xs agressorxs.”

Era comum que a única ação da polícia fosse a de questionar as vítimas e testemunhas pedindo provas de sua identidade e situação legal. Entretanto, em muitas ocasiões quando as comunidades asiáticas começaram a se defender, ficou evidente uma preocupação maior da policia com os incidentes racistas, porém com situações de prisão das vítimas e criminalização assustadoramente comuns.

 

Mas isto não era novidade naquela época: um relatório de 1970 da organização Runnymede Trust já relatava uma situação em que a polícia reagiu apenas quando surgiram rumores de que asiáticxs estavam comprando armas para se defender dos ataques.

Um caso que ficou muito conhecido foi o dos irmãos Virk, que foram atacados por racistas em frente a casa em que viviam em East Ham em 1977. Os irmãos se defenderam, e um dos agressores foi ferido com uma facada. Os Virks chamaram a policia, que quando chegou os prendeu. No julgamento, as testemunhas de acusação eram os jovens brancos que tinham atacado os irmãos, enquanto os advogados de defesa tentavam demonstrar a natureza racista do ataque. Em julho de 1978, os quatro irmãos Sikh de East Ham, Mohinder, Balvinder, Sukvinder e Joginder Singh Virk foram condenados por lesão corporal grave e sentenciados a doze anos e três meses. Este caso teve grande importância no desenvolvimento de organizações de auto-defesa como o Standing Committee of Asian Organizations e o Newham Defense Committee.


Os anos 70 chegavam ao fim, mas os ataques racistas permaneciam comuns. De 1978 a 1980 diversos casos ocorreram, incluindo uma série de assassinatos. Em 1978, um garoto de 10 anos de idade chamado Kennith Singh foi espancado até a morte perto de sua casa em Plaistow, a leste de Londres, e seu corpo foi encontrado em uma pilha de lixo. Os assassinos nunca foram encontrados, assim como na maior parte dos ataques racistas que aconteceram na região. A segunda metade da década também foi marcada pelos ataques frequentes da Anti-Paki League (Liga Anti-Paquistaneses), criada pela gangue Trojan Tilbury Skins, formada por garotos que eram skinheads desde o final dos anos 60 e outros mais jovens. A gangue deixava explícito seu ódio a imigrantes paquistanesxs e asiáticxs em geral, e faziam paki-bashings constantemente em East End, tendo sido apontados como grandes responsáveis por muitos dos ataques à comunidade asiática em 1977.


\subsection{Altab Ali: as comunidades se mobilizam}

Em 1978 a região de Brick Lane passava por muitos ataques, alguns deles com presença de centenas de agressores. Em um deles, 150 jovens brancos atacaram Brick Lane aos gritos de “mate os negros bastardos”, quebrando janelas e para-brisas de carros de lojistas asiáticos. Um deles, Abdul Monam, perdeu dois dentes e levou cinco pontos no rosto após ter ficado inconsciente com a chuva de pedradas que atingiram sua loja.

Em 4 de maio de 1978, Altab Ali, um garoto bengali de 25 anos, estava voltando para casa depois do trabalho na fábrica no East End de Londres quando foi esfaqueado por três adolescentes skinheads – Roy Arnold e Carl Ludlow, de 17 anos, e um terceiro garoto de 16 anos. Deixaram uma mensagem em um muro próximo que dizia “Nós voltamos”.

“Quando Altab Ali foi assassinado,” – relembra o imigrante bengali Sunahwar Ali – “todxs ficaram furiosxs com isso, e muitxs jovens o conheciam porque era uma pessoa local. Desde 1976, muitos ataques racistas estavam acontecendo e se tornou um ponto de virada neste incidente. Na época em que Altab Ali foi morto, todo mundo disse, ‘isso foi o bastante, agora precisamos fazer alguma coisa’, e como resultado conseguimos organizar a manifestação e foi uma das maiores concentrações já vistas na história.”

A morte de Altab Ali foi um momento emblemático na formação e consolidação de organizações de imigrantes, e desencadeou diversas mobilizações da comunidade asiática e movimentos anti-racistas, no que hoje é lembrado como a “Batalha de Brick Lane de 1978”. Dez dias após sua morte, no dia 14 de maio, milhares de imigrantes e anti-racistas marcharam contra a violência racista atrás do caixão de Altab Ali, em uma das maiores manifestações da população asiática na Inglaterra. Centenas de cafés, restaurantes e lojas fecharam em apoio. Grupos como a Action Committee Against Racial Attacks, Tower Hamlets Against Racism and Fascism, Trades Council e a Anti Nazi League ajudaram a reunir e articular protestos e redes de solidariedade. Grupos como o Bangladesh Youth Movement ampliaram suas ações, e nas ruas havia gangues de jovens locais para combater os ataques skinheads. Durante a marcha do funeral surgiram slogans a partir de então muito utilizados pelo movimento, como “Self defense is no offense” (auto-defesa não é agressão), “law and order for whom?” (lei e ordem para quem?), “Who killed Altab Ali? Racism!” (quem matou Altab Ali? O racismo!), “Black and White, unite and fight!” (Negrxs e brancxs, se unam e lutem!), e “Here to stay, here to fight!” (Aqui para ficar, aqui para lutar!).

Rajonuddin Jalal, um dos envolvidos na formação do BYM, lembra que “foi um evento muito emocionante, centenas de pessoas vieram. Nunca antes Bengalis vieram de forma tão numerosa. Emocionante no sentido de que o evento não foi organizado por pessoas que eram profissionais no campo político. Foi uma resposta comunitária espontânea a um acontecimento.”

Ao mesmo tempo em que acontecia a passeata, alguns grupos como a Progressive Youth Organisation decidiram não participar da manifestação para proteger seus bairros, que estando vazios eram um prato cheio para ataques racistas. Como haviam pensado, Brick Lane foi atacada, e os membros do grupo conseguiram confrontar os agressores fisicamente e detê-los. Outros casos assim aconteceram depois.

Cada vez mais, as comunidades vão se organizando tanto no campo político, quanto no nível da auto-defesa de rua. Além das organizações mais amplas e politicamente voltadas a lidar com questões específicas como racismo, educação, moradia, etc., iam se formando pequenos grupos de vigilância que tentavam defender suas comunidades dedicando-se a lutar contra racistas, apoiando também comunidades de outros locais. Segundo Mohammed Abdus Salam, sentiam que deveriam “(…) ser física e intelectualmente resistentes. Também acreditávamos que tínhamos que preparar nossxs jovens para seguir com suas próprias pernas e revidar. Então começamos a fazer treinos físicos e kung-fu. (…) Sabíamos que a polícia não ia nos ajudar e apoiar, a polícia estava sempre do lado das pessoas brancas. Por isso desenvolvemos a tática de bater e correr. (…) Gradualmente nosso grupo se tornou maior e maior”.

Conforme a juventude vai ganhando confiança e começam a partir para a ofensiva, as atividades vão ganhando ainda mais impulso. Nessa época surge a Hackney and Tower Hamlets Defence Committee, que participou de diversas ações como a manifestação em frente à Delegacia de Bethnal Green contra a brutalidade policial e falta de ação contra os ataques racistas; ocupação da esquina entre Brick Lane e Bethnal Green Road reivindicando o fechamento do ponto de vendas da NF de seus jornais e materiais racistas e fechamento de sua sede na área; patrulhas em Brick Lane aos sábados para barrar os encontros de racistas que se reuniam para planejar os ataques dos domingos; organização do Dia de Solidariedade Negra, um dia inteiro de greve em Tower Hamlets contra ataques racistas que levou toda a região a uma paralização; ações de oposição ao plano governamental de construção de guetos segregados para comunidade bengali; entre outras. Toda esta movimentação e os inúmeros confrontos nas ruas contribuíram definitivamente pra que os racistas fossem retirados finalmente da área, e a National Front tivesse sua sede na região fechada.

No verão de 78, grupos anti-racistas e comunidades asiáticas de East London e Southall intensificaram as ações contra a National Front e os ataques racistas, tendo recebido apoio e visitas de líderes de sindicatos e autoridades políticas. “Em 78, após a morte de Altab Ali, – recorda Sunahwar Ali – em um domingo a National Front marchou por Brick Lane e começou a quebrar todas as janelas das lojas. (…) as pessoas gritavam, nós saímos e vimos que a National Front e os skinheads estavam vindo com paus e pedras em marcha. Eram cerca de 25-30 em número. Domingo era o dia de folga e a maior parte de nós estava ali de vigilância para algum ataque ou algo do tipo, estávamos mentalmente preparados para nos defender. Saímos para encará-los, a polícia chegou 5 minutos depois. Eles conseguiram prender alguns deles. Não sei o que aconteceu depois da prisão. Mas na primeira meia hora a polícia não fez nada, enquanto quebravam as janelas da loja bengali. Tivemos que lutar com eles mano a mano. Pensamos, isto é Brick Lane, esta é nossa casa e se não defendermos Brick Lane não podemos viver neste país. Nós tivemos que fazer isso, não havia outra escolha.”

Em junho, uma semana depois de ataques racistas organizados em Brick Lane e East End, a ANL organizou em conjunto com as organizações da comunidade bengali local uma manifestação com presença de mais de 4 mil pessoas, e outras ações foram feitas em contraposição à NF no decorrer de julho. Ainda nesse mês 8 mil trabalhadorxs entraram em greve após um chamado da Hackney and Tower Hamlets Defense Committee por um Dia de Solidariedade Negra. A prisão de dois jovens bengaleses após um incidente em que foram atacados também levou a manifestações com cerca de 3 mil pessoas em frente à delegacia de Bethnal Green em 18 de julho do mesmo ano.

Com o fim da batalha nas ruas de 78, as organizações começam a se estruturar coletivamente. Surgem iniciativas como a Federação Bangladeshi de Organizações da Juventude (FBYO), que envolvia as organizações Bangladesh Youth Movement; Bangladesh Youth Association; Progressive Youth organisation: Bangladesh Youth League Birmingham, Sunderland Bengali Youth Organisation; Bangladesh Youth Approach; Shapla Youth Force; Weavers Youth Forum; Bangladesh Youth League; Bangladesh Youth League Luton; Eagle Youth Organisation; Overseas Youth Organisation; Hackney Bangladesh Youth Organisation, Wallsall Youth Organisation; Bradford Youth Organisation e a League of Joi Bangla Youth, entre outras. Publicaram uma revista bilingue chamada Jubobarta, se envolveram em produção de documentários e passaram a debater e agir nas esferas políticas com relação às necessidades e urgências da comunidade em todas as esferas da vida.


Em 1979 é publicada também a primeira revista da AYM em Bradford, com o nome de Kala Tara, que significava Estrela Negra – uma identificação com as lutas negras. “O racismo é violento e os racistas estão fazendo barulho. A iminente crise econômica irá colocar nossas vidas ainda mais em risco neste país. Cabe à comunidade negra como um todo se erguer e tomar as rédeas da luta contra o racismo”, dizia a primeira página da revista. Kala Tara também divulgou diversas campanhas da AYM, como campanhas contra deportações e separação de famílias, denúncias de políticas racistas de restrição de imigração, fotografias de manifestações públicas e outras atividades.

\section{Parte III}

\subsection{Southall, 1979}

Nas Eleições Gerais de 1979, a National Front emplacou 303 candidatos, o que não só colocava o partido em outro patamar na política nacional, como lhe dava o direito a tempo no rádio e televisão. Em resposta, a ANL e seus apoiadores se apuseram à NF em todos os lugares em que fizessem comícios eleitorais.

Em abril, a National Front realizaria um comício eleitoral no Southall Town Hall, o que foi considerado como uma provocação, já que Southall tinha uma das maiores comunidades asiáticas do país. Um dia antes do comício, cinco mil pessoas marcharam até Ealing Town Hall para protestar contra a NF, entregando uma petição assinada por 10.000 moradorxs. No dia seguinte, 23 de abril, lojas locais, fábricas e transporte público pararam em greve a partir das 13hs, e a rua foi bloqueada. Mais de dois mil policiais foram proteger os apoiadores da NF, em muitos casos escoltando-os até o local. Pouco depois, conforme Balwindar Rana relembra, ‘(…) um ônibus passou, com skinheads dentro, fazendo sinal de V. Alguns dxs jovens asiáticxs começaram a brigar com os skinheads, e a polícia respondeu agredindo xs asiáticxs.’. Com participação de centenas de manifestantes, este ato se tornou uma batalha com a polícia, que teve 40 pessoas feridas e 300 presas. Pelo menos três pessoas tiveram traumatismo craniano, e outras foram espancadas até perderem a consciência. Militantes anti-racistas passaram a noite toda resgatando garotxs asiáticxs que foram presxs pela policia e que, após apanharem, eram jogadxs nos arredores distantes de Londres. 




Blair Peach, um professor que fazia parte da Anti-Nazi League, morreu com ferimentos na cabeça causados pela polícia. Na televisão, políticos e “especialistas” condenavam a população de Southall pelo ocorrido, exibindo apenas imagens de policiais feridos. No sábado, quinze mil pessoas marcharam em memória de Peach, e durante a semana que se seguiu, foram muitas as discussões, panfletagens e atividades.

Seu enterro, no dia 13 de junho, foi acompanhado por dez mil pessoas, e no ano seguinte outras dez mil marcharam por Southall em sua memória. Uma escola recebeu seu nome em homenagem, e outras atividades foram organizadas desde então. Neste contexto há também o surgimento do Southall Monitoring Group, em meio a uma campanha para que os assassinos de Blair Peach fossem julgados. O trabalho principal do grupo era o monitoramento de ataques racistas e atividade policial. Outro grupo que surge em meio as manifestações por Blair Peach é o Southall Black Sisters, que se engaja diretamente na luta contra o machismo e pelo direito de autodefesa da mulher.


\subsection{Akhtar Ali Baig}

Mais um assassinato que teve grande repercussão foi a morte de Akhtar Ali Baig por uma gangue skinhead em julho de 1980, na East Ham High Street. Baig foi parado por dois garotos e duas garotas de 15 a 17 anos e esfaqueado no coração por Paul Mullery, de 17 anos. A polícia tentou descrever o ataque como não racista, mas testemunhas seguiram os skinheads até uma de suas casas permitindo que fossem identificados. Dois deles eram conhecidos por professorxs da Plashet School que já haviam ouvido comentários racistas e presenciado ataques a asiáticxs na escola. A polícia também encontrou material nazista, da National Front e menções contra paquistanesxs no quarto de um dos membros da gangue.

Logo após o assassinato, 150 jovens asiáticxs e afro-caribenhxs marcharam até a Delegacia de Policia Forest Gate, mas a polícia se negou a dar qualquer informação. Neste contexto é formado o Newham Youth Movement, chamando manifestações em massa que levaram a muitas prisões. Em 19 de julho, 2500 pessoas seguiram em passeata por Newham, com mais vinte e nove prisões após conflitos com a policia. A segunda manifestação foi organizada pelo Newham Youth Movement e o Steering Committee of Asian Organizations, e foi apoiada por mais de 5 mil pessoas. A pressão para que o racismo não fosse ignorado era grande, e por fim o juiz concluiu que o assassinato foi “plenamente motivado por ódio racial”. – “Tudo por uma merda de Paki”, disse Paul Mullery no julgamento.

A partir de uma campanha comunitária por justiça logo após seu assassinato, surge o Newham Monitoring Project, com o propósito de monitorar ataques racistas e a resposta das autoridades policiais e locais. Como até mesmo as autoridades locais admitiram em 1981, os ataques racialmente motivados eram muito comuns, atingindo cerca de 60 vezes mais as pessoas negras/asiáticas do que as brancas. Em abril, ocorre uma revolta em Brixton logo após um incêndio fatal, com reivindicações da comunidade de que a polícia investigasse suas causas, já que acreditavam ter sido um ataque racista. A polícia agiu de forma truculenta e fez diversas prisões.


\subsection{As revoltas de 1981}

Cada vez mais, as comunidades asiáticas iam se organizando e nenhuma violência racista acontecia sem resposta. Julho de 1981 é lembrado até hoje pelas diversas revoltas populares que ocorreram no Reino Unido durante o mês. Logo em seus primeiros dias, em 4 de julho, estava marcado em Southall um show com as bandas Oi! The Business, The Last Resort e 4-Skins na Hambrough Tavern, e centenas de skinheads vieram de diversas partes de Londres para ir ao show. Segundo testemunhas da comunidade, no percurso janelas de comércios asiáticos foram quebradas, slogans da National Front pixados, lojistas asiáticxs e outras pessoas foram agredidas nas imediações da Broadway, a caminho do pub. A Southall Youth Movement foi alertada por moradorxs locais, e coordenou uma resposta imediata e massiva, reunindo em pouco mais de uma hora uma enorme quantidade de ativistas e apoiadorxs em frente ao bar. Ante a falta de resposta da policia aos pedidos de que o show fosse parado, xs jovens começaram a forçar fisicamente a saída dos skinheads. Uma enorme batalha com a policia começou, e moradorxs da área quebravam seus próprios muros em solidariedade, passando tijolos para xs jovens e apoiando xs feridxs. Centenas de jovens asiáticxs locais jogaram bombas e incendiaram a Hambrough Tavern, deixando tudo em chamas.

\begin{quote}
Os restos queimados da Hambrough Tavern se tornaram um santuário da resistência da comunidade aos ataques racistas e uma cena de alegria para a comunidade local. O Incêndio na Hambrough Tavern enviou um sinal pelo país de que Southall era uma área imprópria para racistas e de que a comunidade asiática iria se erguer e defender a si mesma.\footnote{Balraj Purewal, da Southall Youth Movement in THE ASIAN HEALTH AGENCY – Young Rebels: The Story of the Southall Youth Movement.}
\end{quote}

Alguns dias depois, o clima de tensão foi aumentando. No dia 9, surgiram rumores da ameaça de ataque a um templo sikh onde aconteceria uma reunião da Anti-Nazi League. A policia pedia aos comerciantes que protegessem suas lojas devido ao eminente ataque de skinheads, enquanto grupos de jovens saíram às ruas para enfrenta-los. No dia seguinte, mais uma vez a policia avisou a comerciantes que cobrissem suas janelas com tábuas, e nas proximidades de Hounslow e Handsworth todo o comércio estava coberto com pedaços de madeira. Em Handsworth, a policia informava que um grupo de skinheads vinha do leste de Bromwich para atacar. Na mesma noite, uma gangue passou por Chapeltown aos gritos de slogans racistas e de torcidas organizadas de futebol, destruindo vidraças de comércios. Em Southall, centenas de jovens asiáticxs, caribenhxs e brancxs saíram em passeata, e um violento confronto aconteceu com a polícia, com destruição de carros e viaturas incendiadas. Em Woolwich, cerca de 300 pessoas saíram às ruas para impedir o ataque, que não aconteceu, acabando em mais um confronto com a polícia. No mesmo dia, em Brixton, um homem foi detido após impedir uma abordagem policial abusiva na rua, gerando novos conflitos.

Em Luton, jovens de diversas etnias caçaram os skinheads que se reuniam no centro da cidade, atacando a polícia que os protegia e destruindo a sede do partido conservador. Havia ali um ativo movimento contra o racismo, e nesta época ocorreram muitos ataques racistas, incluindo um ataque em uma mesquita e um templo Sikh. Em maio uma marcha anti-racista organizada pelo Luton Youth Movement foi atacada por cerca de 30 skinheads racistas. Mais tarde, o mesmo grupo atacou uma mulher asiática e duas crianças na Avenida Pomfret. Durante as revoltas de julho, jovens negrxs, brancxs e asiáticxs de Luton atacaram um bar frequentado por estes skinheads e houve confronto com a polícia no Bury Park. No dia seguinte, rumores de que um festival em High Town seria atacado motivaram uma manifestação que terminou em um conflito violento com a polícia, com coquetéis molotov e muitas prisões.

Em Toxteth, um jovem negro foi acusado pela policia de ter roubado a moto que dirigia. Outro homem negro intercedeu e acabou sendo preso, culminando em confrontos com a policia, barricadas e incêndios nos dias que se seguiram. Quase como num efeito dominó, outras muitas regiões com grande população não-branca foram tomadas por revoltas populares, tomando proporções gigantescas em lugares como Birmingham, Southall, Nottingham, Manchester, Leicester, Southampton, Halifax, Bedford, Gloucester, Wolverhampton, Coventry, Bristol e Edinburgo. Começaram a acontecer saques, incêndios de prédios, destruição de carros. Posteriormente a mídia e demais discursos acabaram se focando muito mais em críticas relacionadas a vandalismo e comparações vazias com rebeliões anteriores em Brixton. Assim, em geral foram ignoradas as origens destas revoltas na ameaça de ataques racistas, na indiferença policial e criminalização das vítimas, e uso da lei SUS (Stop and Search) – que permitia a detenção sem provas de pessoas com atitude considerada suspeita –de forma injusta e racista.

\subsection{Bradford 12}

\begin{quote}
Tendo em vista os acontecimentos recentes em Southall, Londres e outras áreas onde famílias negras foram atacadas com bombas e assassinadas. Onde a juventude negra assim como em Depford tem sido espancada até a morte, recebemos a notícia que ônibus cheios de skinheads estavam vindo para Bradford muito seriamente. Muitas pessoas em Bradford de todas as esferas de vida estavam falando abertamente desta forma. Acredito que quando as pessoas são atacadas é seu direito agir em auto-defesa. A natureza desta defesa depende da natureza do ataque e dos agressores. A defesa do povo negro ou todxs xs trabalhadorxs e pessoas que são ameaçadas pelo fascismo necessitam de organizações de defesa: é com isto em mente que fizemos o que fizemos.\footnote{Declaração de Tariq Mehmood Ali para a \textit{West Yorkshire Police}, 31/7/81.}
\end{quote}

Em meio às revoltas, em 11 de julho de 1981, 12 jovens da United Black Youth League foram presos em Bradford, acusados posteriormente de posse de explosivos e conspiração. Tudo começou a partir de rumores de que a National Front ou os skinheads estariam indo à Bradford, e a polícia pediu que todos ficassem em casa. Conforme Tariq Mehmood relembra, a comunidade decidiu agir: “(…) achamos que isto era totalmente errado. Não iríamos ficar em casa, a gente ia sair e organizar as pessoas.” No final de semana anterior já tinham acontecido ataques em outras cidades e a resposta da polícia mostrou a necessidade de que as comunidades se defendessem por si próprias. “(…) tomamos a decisão de que não deixaríamos uma situação similar aconteceria em Bradford, com fascistas andando e destruindo a parte de Bradford onde as comunidades negras viviam.”, disse Saeed Hussain. Prepararam bombas e molotovs para caso o ataque realmente acontecesse, mas os fascistas não atacaram Bradford e pouco depois se soube das prisões.

Este caso gerou enorme repercussão, e em uma reunião logo em seguida entre pessoas da comunidade negra e asiática e membrxs da esquerda branca, decidiu-se pela criação de uma campanha de solidariedade que tomaria grandes proporções: Bradford 12.

Houve piquetes contínuos na prisão e na audiência, criação de comitês de apoio em outras localidades, panfletos e cartazes em diversas línguas, e mobilizações com grande participação de pessoas. “Não há nenhuma conspiração, além da conspiração policial”, dizia um dos slogans.

Giovanni Singh, Praveen Patel, Saeed Hussain, Sabir Hussain, Tariq Ali, Ahmed Mansoor, Bahram Noor Khan, Tarlochan Gata Aura, Ishaq Mohammed Kazi, Vasant Patel, Jayesh Amin e Masood Malik se apresentaram no Tribunal de Leeds em 26 de abril de 1982, e o julgamento durou 31 dias. Por fim, a comunidade ganhou o julgamento, como relembra Saeed Hussain: “Sim nós realmente ganhamos o julgamento, mas a vitória real foi que as comunidades negras realmente demonstraram que tinham o direito de defender a si mesmas. E acho que isso foi levado pra outras partes do país também.”


\subsection{Os anos 80 prosseguem…}

Os ataques racistas prosseguiram no decorrer da década, com assassinatos, espancamentos e perseguições. Em meados dos anos 80, os ataques denunciados contra a comunidade negra e asiática haviam mais do que dobrado. Em 1987 estimava-se que uma em cada quatro pessoas não-brancas haviam sido vitimas de ataques racistas.

Nas escolas, a situação era bastante problemática, com tentativas sistemáticas de infiltração da extrema direita junto à juventude. Já no início da década se intensificam as atividades da National Front e outros partidos e grupos com panfletagens, recrutamento, distribuição de literatura racista e outras ações em escolas, com um crescimento alarmante da infiltração fascista e dos esforços de cooptação da juventude branca.

Desde 1977, a Young National Front, braço da NF, passa a atuar focada na juventude, como uma das primeiras organizações a lançar campanhas em escolas. A YNF também difundia o boletim Bulldog, editado por Joe Pearce, e suas investidas com grupos de jovens e bandas musicais de caráter nacionalista rendeu a aproximação com Ian Stuart e a banda Skrewdriver, que seriam peças chave na consolidação de skinheads White Power, abertamente neonazistas, na Inglaterra.

A violência racista nas escolas e playgrounds também ganhava maior impulso, e surgiriam inclusive organizações de professorxs brancxs com o intuito de dar aulas particulares para as crianças brancas. Neste momento já existem movimentos como a All London Teachers Against Racism and Fascism (ALTARF), com monitoramento dos incidentes racistas e atividades de combate.

Entre os ataques racistas e atuação política da extrema-direita, os esforços de recrutamento fascista na juventude branca, as investidas do Estado com legislações de imigração, deportações, racismo policial e problemas econômicos, a situação das comunidades imigrantes era muito difícil. Tudo isso, porém, não acontecia sem que houvesse resposta das comunidades asiáticas e negras organizadas com apoio de movimentos anti-fascistas. Houve novos casos de prisões e repressão policial após situações de auto-defesa como o caso de Ahmed Khan em 1982 e Newham 7 em 1984 – que gerou uma intensa campanha de solidariedade. Houve também muitas campanhas contra deportações e legislações de imigração, ações de apoio às lutas anti-apartheid na África do Sul e da causa Palestina, publicações de revistas, jornais e materiais informativos, solidariedade a greves operárias, apoio jurídico e físico à vítimas de ataques, e atividades diversas ligadas ao incentivo da auto-defesa e combate ao racismo.

 


A crescente organização das comunidades que, desde 1976, tomava grande impulso, criou novos contextos e possibilidades de luta. O mito dos povos asiáticos como pessoas passivas e tímidas havia sido quebrado, e a juventude mostrou que estava ali para ficar, e que lutariam até o fim por suas comunidades.

\begin{quote}
As organizações negras da cidade, particularmente a comunidade Bangladesh e a AYM, mostraram como a resistência unida pode ser efetiva. (…) Agora é hora de nos livrarmos de todas as hesitações que temos. Precisamos nos unir. Precisamos defender os direitos de nossa própria comunidade (…) Nosso direito a permanecer aqui e nosso direito de lutar contra o racismo.

Boletim Kala Mazdoor \#1, Asian Youth Movement Sheffield
\end{quote}


\section{Bibliografia}

\begin{Parskip}
ALTAB ALI FOUNDATION – Commemorating Altab Ali Day 4 May against racism and fascism

ASIAN YOUTH MOVEMENT BRADFORD – Kala Tara \#1

ASIAN YOUTH MOVEMENT MANCHESTER – Liberation

ASIAN YOUTH MOVEMENT SHEFFIELD – Kala Mazdoor \#1

BOWLING, Benjamin – Violent Racism: Victimization, Policing and Social Context

CAMPAIGN AGAINST RACISM AND FASCISM BOLETIM – Racist Activity in Schools

GORDON, Paul – Racial Violence and Harassment

KERSHEN, Anne – Strangers, Aliens and Asians: Huguenots, Jews and Bangladeshis in Spitalfields 1666-2000

LONG, Lisa – Revolution and Bloodshed: Representations of African Caribbeans in the Leeds

MANZOOR, Sarfraz – Black Britain’s darkest hour

PROCTER, James – Writing Black Britain 1948-1998: An Interdisciplinary Anthology
RAMAMURTHY, Anandi – Secular Identities and the Asian Youth Movements

RAMAMURTHY, Anandi – Kala Tara: A History of the Asian Youth Movements in Britain in the 1970s and 1980s

REES, John \& GERMAN, Lindsey – A People’s History of London

TEARE, Keith – Under Siege: Racism and Violence in Britain Today

THE ASIAN HEALTH AGENCY – Young Rebels: The Story of the Southall Youth Movement

THE RACE TODAY COLLECTIVE – The Struggle of Asian Workers in Britain Reflecting on the Trial of the Decade: The Bradford 12

WITTE, Rob – Racist Violence and the State: A Comparative Analysis of Britain, France and the Netherlands

WOOLEY, Simon – Ugandan Asians in Britain. Operation Black Vote

The Southall History | http://www.thesouthallstory.com

Swadhinata Trust | http://www.swadhinata.org.uk/

Tandana – Glowworm | www.tandana.org
\end{Parskip}


\part{Urgência das ruas: embates antifas no presente}

\chapter{Não começou ontem, não vai terminar com as eleições}

\hfill{}\textsc{imprensa marginal}

\bigskip

Nos últimos dias, Mestre Moa morreu após levar DOZE facadas em uma discussão com um eleitor de Bostonazi. Uma garota foi torturada pela polícia após ser pega escrevendo \#elenão no muro perto de sua casa. Outra garota que andava na rua com uma camiseta escrito \#elenão foi agredida e teve uma suástica marcada a faca em sua barriga. Um senhor teve sua receita rasgada pela médica que o atendia quando disse a ela que seu voto não era para Bostonazi. Outras tantas pessoas foram agredidas verbal e fisicamente, perseguidas e alvo de diversos tipos de violência nos últimos dias pelos mesmos motivos.

Mas a violência no Brasil não para no processo eleitoral de 2018. Há 518 anos indígenas e negrxs são alvo de um verdadeiro genocídio; a polícia mata diariamente nas periferias; mais de uma centena de mulheres é estuprada todos os dias no país; lésbicas, gays, pessoas trans e bissexuais são mortxs, humilhadxs, agredidxs – lembrando que o Brasil é o país onde mais se mata pessoas trans no mundo inteiro; ativistas de movimentos sociais são perseguidxs e muitas vezes assassinadxs; grupos neonazis espancam, atacam e matam pessoas nas ruas; mulheres morrem aos montes em abortos clandestinos; os problemas sociais são tratados como caso de polícia e a população é massivamente encarcerada em verdadeiros depósitos de seres humanos. Há 518 anos somos exterminadxs, e reina a lógica racista, misógina, LGBT-fóbica.

E enquanto alguns homenageiam antigos torturadores da ditadura militar como se fossem heróis, poucos se lembram das tantas Cláudias, Marieles, pessoas mortas nos Crimes de Maio, Massacre do Carandiru, Eldorado dos Carajás, e tantas outras chacinas que ocorrem diariamente pelo país que em grande parte das vezes terminam com condecorações para a policia ou simples afastamentos temporários.

Falta pouquíssimo tempo para o segundo turno das eleições e temos visto muitas discussões e debates entre anarquistas, feministas, antifascistas e libertárixs em geral com opiniões bastantes diferentes sobre o que fazer. Esse processo eleitoral, mais ainda do que todos os outros anteriores, tem se mostrado uma grande novela de mau gosto, com episódios pautados em todo tipo de jogo sujo. Temos, obviamente, o ponto em comum de que ninguém vê com bons olhos a possibilidade da chegada de Bolsonazi a presidência do Brasil, com todo seu racismo, machismo e homofobia escancarados e defesa de políticas de extrema direita, ditadura militar, tortura, etc.
O ponto em desacordo muitas vezes tem sido a possibilidade do uso do voto como ferramenta estratégica para não permitir que ele seja eleito, ou, por outro lado, a estratégia desde sempre cara a muitxs anarquistas da campanha pelo voto nulo ou abstenção como meio de trazer a tona o caráter ilusório das urnas como meio de mudanças sociais reais, e a propaganda pela criação de organizações populares horizontais. Em meio a isso surge também todo esse processo de mulheres se organizando em dezenas de cidade e indo as ruas massivamente contra Bolsonazi – e que bom ver as pessoas indo às ruas.

Porém, uma coisa é fato. Tudo isso que se está vivendo na atualidade no Brasil está muito além de Bolsonazi, que de certa forma é só a parte mais visível de um problema muito maior, e que também começou muito antes de que este se tornasse o ícone midiático que é hoje. Por mais heterogenea que seja sua base eleitoral, é nitidamente visível pelas estatísticas a presença entre seus apoiadores de um número enorme de jovens. Desse número, obviamente há aquelxs que, fechando os olhos para tudo de retrógrado que representa, vêem erroneamente Bolsonazi como uma suposta ‘solução’ à desesperança com a atual situação política e o descrédito que a esquerda partidária caiu, a partir da associação no senso comum entre esquerda e corrupção, mas também a partir de anos de políticas que procuraram amansar movimentos sociais, alianças esdruxulas pautadas por uma suposta governabilidade e políticas onde por fim quem dita as regras é a lógica do capital e das grandes corporações mundiais, com seus interesses que obviamente nada tem a ver com os nossos.

Mas dentro de tudo isso é preocupante o crescimento exponencial de um outro tipo de mentalidade entre a juventude – aquela que flerta realmente com a extrema direita, o ideário fascista, as declarações racistas, misóginas e homofóbicas do candidato. É assustador como nesse quadro, pessoas se sentem cada vez mais a vontade – e com apoio – para agir violentamente, agredir, vomitar idéias fascistas abertamente, e fazer valer de todas as formas sua intolerância. Se antes denunciávamos a ação fascista violenta de pequenos grupos como skinheads white powers, separatistas, integralistas e outros dessa corja, agora vemos vizinhos, parentes, pessoas conhecidas ou nem tanto, que defendem Bolsonazi e suas declarações absurdas, defendendo golpes e intervenções militares, ações truculentas da polícia, pena de morte, e por aí vai. Mas isso não é obra de Bolsonazi, por mais influência que tenha no momento. É um processo que surge antes ainda com múltiplos fatores, e que encontra ele como rosto visível nesse momento, contribuindo para que se amplie. Mas a verdade é que nos próximos tempos, com ou sem Bolsonazi, podem surgir novos ícones, novos rostos, novos porta-vozes. E a era da internet e das redes sociais ajudam muito nisso. Esse tipo de mentalidade tem se instalado e espalhado na juventude que, cansada do que existe, tem se alinhado à direita de forma extremamente preocupante. E seja lá qual for o resultado das eleições, precisamos pensar nisso com muita seriedade.

Tempos estranhos nos esperam, e cabe a nós anarquistas, feministas, antifascistas, analisar essa conjuntura com cuidado, procurar entender como se deu e tem se dado esse processo, quais nossas falhas durante esses anos todos, e como lidar com a situação para que seja possível seguir para outros rumos. Para além de Bolsonazi e das eleições que se avizinham – e que o “coiso” não chegue ao poder – , que soluções reais podemos criar e colocar em prática coletivamente para fazer frente a isso? Nossas discussões e ações coletivas precisam ir além!

Essa luta não começou ontem, nem vai terminar com o resultado das eleições. Essa luta está para além das urnas ou de um período eleitoral, ela deve seguir nas ruas, todos os dias.

\chapter*{Somos todos antifa, menos a polícia\subtitulo{Sobre como e com quem lutamos}}
\addcontentsline{toc}{chapter}{Somos todos antifa, menos a polícia}

%Revisar itálicos do original https://faccaoficticia.noblogs.org/post/2019/09/07/antifascistas-vs-policia/#sdfootnote2anc

\hfill{}\textsc{facção fictícia}

\bigskip

\section{A REAÇÃO BATE À PORTA}

Entramos com tudo em um tempo de reação. A década progressista dá lugar a uma onda de movimentos e governos de extrema direita ganhando espaço em todo o mundo. É difícil acreditar que existe alguma surpresa nisso. Como poderíamos nos surpreender com a eleição de Trump nos EUA e Bolsonaro no Brasil, “quando Putin, Berlusconi, Erdogan, Modi e Netanyahu têm reinado por anos no mesmo modelo”\footnote{In The Name of The People, LIAISONS.} na Rússia, Itália, Turquia, Índia e Israel?

Estados Unidos e Brasil são os retardatários em uma tendência mundial de governos de direita chegando ao poder democraticamente. Trump e Bolsonaro não são fascistas se usamos a palavra com rigor histórico e uma análise apurada de suas influências e características políticas. No entanto, ambos mobilizam emoções e ressentimentos comuns ao fascismo presentes em grande parte das camadas populares, e também das classe média branca e elites conservadoras que historicamente se beneficiam de privilégios desde a época da colonização e da escravidão institucionalizada nas Américas. Eles falam para os que se sentiram “esquecidos” pelas políticas sociais de programas de governo da última década, como o caso dos democratas de Obama nos EUA, e o PT de Lula e Dilma no Brasil. Portanto, entendemos os governos de Trump e Bolsonaro como populistas de extrema direita. Eles buscam aplicar reformas e ataques a direitos sociais conquistados para reinventar uma forma de governar “em nome do povo”. Sobretudo, são governos que se mantém a forma democrática, mas praticam a violência de Estado buscando promover a segurança, são, portanto, democracias securitárias.

\begin{quote}
Estejam eles vindo de raízes ‘populares’ ou apenas apropriando seu estilo, esse grupo [de governantes] exuma aquela chamada aliança entre o soberano e seu ‘Povo’. Eles criam a aparência de um abismo no outro lado onde as elites buscam refúgio, espremidas juntas sob a obscura luz do ‘deep state’. Esse novos populistas ganharam corações com a promessa de salvaguardar tudo o que, em nome do povo, é idêntico a eles mesmos, a fim de fazê-lo se levantar, em uníssono, contra a ameaça das minorias étnicas, sexuais ou políticas – um gesto que muitas vezes parece se estender ao ponto de incluir, em um momento ou outro, quase todo mundo. Das entranhas destas massas que vagam longamente no deserto neoliberal, elas ressuscitam um novo Povo de ressentimento.

Liaisons, \emph{In The Name of The People}
\end{quote}

\section{A VIOLÊNCIA NÃO ACABA, MAS É DIRECIONADA CONTRA AS MINORIAS}

Nenhum estado democrático reprime ou elimina definitivamente as milícias ou grupos fascistas e racistas. No Brasil não foi diferente: em 1964 vivemos um golpe de estado com armas, tanques e disposição para matar, torturar e fazer sumir milhares de pessoas. Em 2018, vimos os herdeiros do aparato militar ditatorial, que foi para o crime organizado das milícias durante a era democrática, organizarem a vitória eleitoral de seu patrono. E Jair Bolsonaro não tem nenhuma vergonha em elogiar e estimular ações ilegais como a tortura e o extermínio, seja de suspeitos de cometer algum crime ou povos originários habitando uma terra que é sua desde muito antes. E é nessa área cinza entre o legítimo e o ilegítimo, entre a violência policial legalizada e a agressão criminosa de gangues e milícias, que o fascismo opera e cresce para, quando tomar o controle do Estado, poder usar sua força total através de grupos de extermínio, das polícias e das prisões e campos de concentração mantidos e expandidos nos períodos democráticos.

equer diminuir a gigantesca violência necessária pra manter o Capitalismo neoliberal em sua fase decadente e de crise permanente. O que ele pretende é canalizar essa violência o máximo possível para as minorias políticas: as populações negras, LGBTTTIQ, mulheres, indígenas, imigrantes e pobres. A imagem do “cidadão de bem” que quer ser protegido pela liberação do porte de armas é a imagem do homem branco, de classe média ou alta e heterossexual, que diz querer defender sua família e seu patrimônio da criminalidade, mas se sente muito mais ameaçado politicamente pela ascensão de membros das classes subalternas, pela liberdade das mulheres e de pessoas não heterossexuais ou praticam sexo de forma dissidente. Os que mais se beneficiam diretamente da política de liberação de armas serão os mesmos ruralistas que já praticam torturas e assassinatos nos campos e as milícias que controlam bairros e municípios inteiros em cidades como o Rio de Janeiro. Para o senhor presidente, violência se combate com medidas que apenas aumentam a violência classista, racista e sexista no país.

Para canalizar essa violência contra as minorias, esses líderes precisaram deixar claro seu projeto para serem eleitos. Bolsonaro e Trump não foram eleitos apesar de serem abertamente sexistas, racistas, homofóbicos. Eles foram eleitos justamente porque são tudo isso. E não apenas o presidente, mas vários parlamentares foram eleitos pela mesma lógica. O candidato Rodrigo Amorim, quebrou a placa em homenagem à Marielle Franco em 2018, enquanto fazia campanha para ser deputado estadual no Rio de Janeiro. Amorim foi eleito como candidato mais votado. Depois de eleito, o deputado emoldurou e pendurou a placa quebrada em seu escritório e alega que estava “restaurando a ordem” quando a quebrou. Para seus eleitores, o fato dele afrontar publicamente a memória ou qualquer homenagem a uma mulher negra, lésbica, criada na favela e que foi assassinada por policiais, é apenas mais uma “demonstração de caráter” de seu candidato.

Quando analisamos esses perfis e suas ações, concluímos que de nada adianta acusar esses políticos de serem machistas, sexistas ou mesmo fascistas. Isso não fará com que percam apoiadores porque foram essas características que atraíram seus apoiadores. A melhor reposta que podemos dar é saber enfrentá-los mostrando que sua política é apenas mais do mesmo, que serão incapazes de melhorar a vida das pessoas dentro do neoliberalismo e entregarão às pessoas apenas mais frustração. Precisamos mostrar que eles são fracos e ainda mais limitados que a organização e solidariedade entre as pessoas.

\section{SERIAM OS POLICIAIS NOSSOS ALIADOS? – E PORQUE POLÍCIA ANTIFASCISTA É UM CONTRASSENSO}

Percebemos, assim, que vivemos em um tempo no qual ideias e emoções fascistas desfilam sem muito receio de se mostrar explicitamente, tentando ganhar propulsão com discursos canalizam o ódio contra as minorias. Por vezes, com novos nomes, como Alt-Rigth (Europa e EUA) ou bolsonarismo (Brasil), mas com as mesmas práticas de eliminação e extermínio das formas de vida que ele declara como insuportáveis e indignas de viver. Hoje, esse fascismo não apenas se serve da democracia, como aprendeu a se perpetuar com uma renovada retórica democrática associada ao desejo por segurança. Eles sabem que as instituições democráticas, ao fim, os favorecem.

Para ficar em um exemplo rápido (e cinematográfico) sobre como as instituições na democracia favorecem o fascismo, assistam o filme “In the fade”, de Fatih Akin, vencedor em Cannes de melhor filme estrangeiro em 2018. No filme, como na vida, a polícia e o tribunal ficam do lado dos neonazistas, sejam eles alemães do PEGIDA ou gregos do Aurora Dourada. Assim acontece qualquer gangue fascista ou neonazista sob o governo de um Estado em qualquer lugar do planeta. Fascismo e Estado democrático de direito não são, necessariamente, antagônicos. E hoje isso é uma verdade por demais evidente.

No Brasil, desde que o bolsonarismo tomou forma político-eleitoral e caminhou em direção à ocupação do governo do Estado por meio da democracia, a temática do antifascismo se espalhou por vários grupos sociais e indivíduos gerando imagens, memes em mídias sociais, camisetas, adesivos, declarações inflamadas etc. É com alegria que os anarquistas, dedicados à lutas antifascista desde sempre, veem isso. Mas essa alegria não abafa a desconfiança de que essa “onda antifa” em uma esquerda mais ampla, seja apenas isso: uma onda; ou pior, uma nova grife, uma identidade ou uma tática de frente única para conter os que são vistos como radicais.

Nesse sentido, é salutar recordar o alerta do coletivo catalão Josep Gardenyes em seu libelo \emph{Uma Aposta para o Futuro} (Edição Subta, 2015, pp. 19-20), que diz o seguinte: 

\begin{quote}
Insistimos na ideia de que o antifascismo é – e tem sido desde os anos 1920 – uma estratégia da esquerda para controlar os movimentos e frear as lutas verdadeiramente anticapitalistas. Ele também sempre foi um fracasso se o pensarmos como uma luta contra o fascismo. As [históricas] estratégias propriamente anarquistas para combater o fascismo foram muito mais efetivas, porque entendiam o fascismo como uma ferramenta da burguesia – e nesse sentido, da democracia –, e dessa forma eles atacaram diretamente o fascismo não no ponto onde ele entrava em conflito com a democracia (direitos, liberdades civis, moderação), mas onde ele convergia com os interesses de proprietários e governantes. (…) O totalitarismo do sistema-mundo atual é uma tecnocracia (…) ele é totalmente compatível com a democracia e não tem nenhuma necessidade de carismas nem de aliança conscientes nem pactuadas entre classes, com seus protagonistas indispensáveis e atores proativos.
\end{quote}

O alerta é, no mínimo, pertinente.

Não queremos com isso dizer que os anarquistas possuem o monopólio da luta antifascista, nem tampouco desprezar ou subestimar a atual onda neofascista e pertinentes reações que ela provoca em amplos setores da sociedade. O alerta provoca uma análise apurada em dois sentidos. Primeiro, é preciso compreender as formas do fascismo contemporâneo e como elas conseguiram equacionar sua presença nas democracias hoje, diluindo as lutas antifascismo no pluralismo democrático e neutralizando seu caráter antissistêmico. Segundo, que ao tomar o antifascismo como principal atividade, os anarquistas correm o risco de cerrar fileiras com aqueles que, mais cedo ou mais, se voltarão contra os anarquistas. Os exemplos históricos são inúmeros, não iremos repetir aqui. Como versa um velho jargão militante: mais importante do que saber contra quem lutamos é saber com quem lutamos. Ao que acrescentamos: mais importante que saber o que fazer, é saber como fazer. A nossa luta já é a vida anarquista em ação.

Mesmo admitindo que uma frente, o mais ampla possível, seja importante para se combater o neofascismo, causa, no mínimo, estranhamento que agora temos que presenciar fenômenos bizarros como o surgimento dos chamados “policiais antifascistas”. Segundo reportagem veiculada pela revista Época, o movimento surgiu em setembro de 2017, composto por policiais civis e militares e demais profissionais da área de segurança pública. Um de seus criadores, um investigador da polícia civil, diz que o Policiais Antifascismo “busca discutir novas políticas de segurança inserindo o policial no debate público — inclusive no que diz respeito aos seus direitos”. A mesma matéria, informa que o movimento conta “com 10 mil membros e representações nos 26 estados brasileiros e no Distrito Federal.”\footnote{Ver https://epoca.globo.com/a-direita-nos-considera-caes-de-guarda-a-esquerda-diz-coisas-que-nos-massacram-23696416.} O cerne das reivindicações do movimento é a crença de que pode haver uma polícia que respeite as liberdades civis e os direitos humanos e que os policiais devem ser vistos e se entenderem como trabalhadores, assim como o são diversos profissionais de outras áreas. Não duvidamos aqui das boas intensões das pessoas, mas não há um só motivo para acreditarmos nessa histórica instituição de opressão.

A polícia emerge, modernamente no século XIX, como um dispositivo de segurança destinado ao cuidado da população. Na antiga Prússia ela surge como medicina social; na França como instrumento das reformas urbanas como resposta às sedições dos trabalhadores; na Inglaterra aparece vinculada à medicina do trabalho e ao controle dos operários nas fábricas, além de sua faceta de proteção à propriedade do comércio marítimo. Na América do Norte, a polícia é herdeira direta das patrulhas de caça e captura de escravos fugitivos. Então, além de sua faceta repressiva contemporânea, a polícia é, desde seu início, um instrumento de governo voltado ao processos de normalização biopolíticos, como mostram as pesquisas de Michel Foucault e Jacques Donzelot. Sua forma ostensiva é mais recente e ao sul do equador foi acrescida de tecnologias de caça e controle coloniais e escravocratas. Nesse sentido, não é exagero dizer que sob qualquer regime político, a polícia é destacamento dos estados dedicado a manutenção da supremacia racial branca, do controle da classe trabalhadora, da imposição de desigualdade material e do patriarcado: todos os valores e requisitos necessários a um estado fascista. E hoje em dia, após o avanço do neoliberalismo desde os 1970, não apenas do Estado, mas de empresas de segurança privada e do desejo de cada cidadão que clama pelo morte do que lhe é insuportável, atuando como um cidadão-polícia.

Assim, quando uma das lideranças do movimento diz, na mesmo entrevista, que “o policial é um garantidor de direitos”, ele não está dizendo nada além da histórica função desse peculiar dispositivo de segurança. Ele segue, justificando a existência do grupo: “a própria palavra polícia significa ‘gestão da polis’. Ele [o policial] deve atuar na cidade garantindo direitos. Ele tem que entender que os direitos básicos de um cidadão são os direitos humanos e fundamentais: o direito à vida, à liberdade de expressão”. Essa declaração expõe, mesmo que involuntariamente, a vinculação da atividade policial com o dever de manter o cidadão e os grupos sociais atrelados ao Estado. Depreende-se disso que, na contingente e elástica atuação cotidiana, cada policial é um agente do golpe de Estado cotidiano que impede que se rompa o vínculo subjetivo, operado nas ditaduras e nas democracias, entre sujeito e governo de Estado. Basta reparar que em todas revoluções modernas, desde a Revolução Francesa e as Independências dos EUA e do Haiti, a única constante invariável é a permanência da polícia – ao lado das prisões, dos exércitos, dos tribunais, das fronteiras. É possível ser antifascista sendo operador de algum destes dispositivos?

\begin{quote}
A polícia não é o oposto dos fascistas. Eles abusam, sequestram, prendem, deportam e assassinam mais pessoas de cor, mulheres e LGBTTTIQ todos os anos do que qualquer grupo fascistas. Eles trabalham mais para fazer avançar a agenda supremacista branca do que qualquer organização de extrema direita independente.

CrimethInc., \emph{What they can’t do with badges, they do witch torches.}
\end{quote}

Enquanto anarquistas, sempre tentamos deixar óbvio que o papel da polícia é impor e reforçar os desequilíbrios econômicos entre as classes, mantendo os pobres sob controle e o patriarcado e a supremacia branca operando como barreiras à igualdade no Capitalismo.

A violência policial não é um caso isolado, uma aberração local ou a característica de um determinado tipo de regime, mas um elemento fundamental para uma sociedade baseada nos direitos de propriedade privada e na autoridade centralizada do Estado. O papel da polícia é manter as desigualdades de classe, raça, gênero e nacionalidade. Eles vão garantir que as pessoas pobres continuem na pobreza, que as excluídas continuem na exclusão, e que as injustiçadas convivam com a injustiça.

Sendo assim, a polícia nunca será uma aliada porque ela é a maior inimiga de quem questiona a ordem imposta, de quem quer mudanças sociais, de quem quer uma vida sem as desigualdades criadas pelo Capitalismo e pelo Estado. Afinal, eles são os primeiros a aparecer para o conflito quando nos cansamos de apenas sofrer as misérias desse sistema e partimos para a ação.

\section{UMA VIDA SEM FASCISMO É UMA VIDA SEM CAPITALISMO, SEM ESTADO E SEM POLÍCIA}

\epigraph{Nenhum governo do mundo combate o fascismo até suprimi-lo. Quando a burguesia vê que o poder lhe escapa das mãos, ela recorre ao fascismo para manter o poder de seus privilégios}{Buenaventura Durruti, em entrevista ao jornalista Van Passen, 1936}

O papel da polícia e o das gangues fascistas não são conflitantes entre si, são complementares. Em 2011, a primeira demonstração pública em defesa das posições do então deputado Jair Bolsonaro foi organizada por skinheads neonazistas em São Paulo. Na época, Bolsonaro era apenas mais um membro desconhecido do parlamento, visto como uma piada, dando declarações racistas e homofóbicas para atrair atenção com polêmicas e escândalos. Dezenas de antifascistas compareceram para impedir que uma marcha neonazi conseguisse ainda mais atenção para Bolsonaro e a polícia ficou entre os dois grupos para impedir um confronto. Quando estamos em grande número, a polícia fica entre nós e os fascistas para “garantir a segurança de todos”. Mas quando somos minoria, os policiais deixam que os fascistas nos ataquem.

Normalmente, a polícia ataca, prende, tortura e mata com impunidade legal. Eles não existem para impedir o crime, mas para garantir que a impunidade para atos considerados criminosos continuem sendo monopólio de quem tem poder econômico e político nas mãos. Nas melhores hipóteses, suas limitações são meramente burocráticas: quando a prisão não é em flagrante e é impossível forjar as provas; ou quando é necessário um mandado judicial para desalojar violentamente um imóvel ocupado; ou então quando uma manifestação popular toma as ruas de forma radical e a violência necessária para contê-la é ilegal ou controversa demais para ser praticada de forma explícita pelas forças policiais. Nesses casos, a ação de bandos neonazistas é útil para fazer o trabalho sujo que a polícia não quer ou não pode fazer num determinado momento.

Uma outra utilidade para a ação fascista nas ruas é nos manter ocupados demais tentando evitar que as coisas fiquem “ainda piores” e para lutar contra o sistema em si. O mesmo acontece com políticos como Bolsonaro e Trump: seus escândalos e suas medidas absurdas nos obriga a estar sempre reagindo às suas agendas invés de seguir as nossas próprias. Isso faz parecer que tudo o que queremos é restaurar alguma “normalidade” perdida no sistema democrático. Passamos a ser apenas defensores da última versão menos absurda da vida sob Capitalismo. O que é sempre o risco de soarmos como reacionários enquanto a direita se apresenta como “os rebeldes antissistema”.

\begin{quote}
…parece que ocorreu uma inversão: por um lado, os progressistas se voltam para o passado, querem evitar a “decadência” dos valores democráticos, e assumem uma posição reativa (que era desde o século XIX a posição dos conservadores clássicos, dos teóricos da decadência etc.). Por outro lado, os populistas de direita, isto é, os reacionários, se tornaram “progressistas” no sentido de que querem acelerar o tempo e adiantar o futuro – mas por isso são apocalípticos. Apocalípticos porque amigos do apocalipse, porque eles não têm pudor em acelerar o processo de devastação do meio ambiente, em aniquilar pessoas (ou simplesmente deixar morrer, como no caso italiano em que impediram que um barco de refugiados atracasse) e em transformar a sociedade em uma guerra de todos contra todos em que sobrevive o mais armado – e isso não é nenhum “retorno à Idade Média”, é o próprio ápice do desenvolvimento capitalista, cuja verdade não é nenhuma versão democrática e luminosa de sociedade, mas sim esse grande Nada destrutivo.”

Felipe Catalani, A decisão fascista e o mito da regressão: o Brasil à luz do mundo e vice-versa
\end{quote}

Se, depois de toda essa reflexão, alguém ainda acredita que se aliar a membros da polícia em alguma luta social revolucionária pode ser uma boa ideia, afirmamos que abrir as portas e confiar em agentes da repressão estatal que querem lutar contra o fascismo é expor nossos movimentos à infiltração e outros riscos extremos desnecessariamente. Após séculos de luta das classes trabalhadoras e excluídas sendo perseguidas, traídas, mortas e aterrorizadas por instituições como a polícia e o exército; e com a sombra de uma ditadura civil-militar ainda viva na memória, é difícil pensar que tais indivíduos possam ser confiáveis – ou que seus colegas o sejam. Deveríamos trazer para dentro de nossas reuniões, protestos e ações, as pessoas que convivem e compartilham o dia de trabalho com assassinos, torturadores e inimigos da liberdade? Se policiais acreditam que todos devem se opor ao fascismo ou a qualquer forma de opressão, seu caminho deve ser o mesmo de qualquer pessoa à frente de instituições repressivas ou exploradoras: desertar. Que abandonem seus cargos, seus salários, seus privilégios e expropriem o máximo de recursos e munições possíveis que devem estar em mãos revolucionárias – e mesmo assim, é possível que levemos anos ou décadas para sequer começar a dar alguma confiança a pessoas que abriram mão de toda decência humana para aceitar um salário em troca de perseguir, prender e matar.

A luta antifascista entre anarquistas é a recusa ao fascismo, mas também é a afirmação da vida. Não podemos e não queremos estar ao lado de quem opera dispositivos de governo. Nesse sentido NÃO somos todas antifascistas, se nos juntamos a uma instituição criada para impedir que as pessoas transformem sua opressão em revolta.

Por essas e outras, os anarquistas sempre tiveram claro que não existe luta antifascista no interior da instituições. Derrotar o fascismo significa obstruir sua virtualidade contida em qualquer Estado, em especial nas instituições que racionalizam e operam o extermínio: a polícia, o exército, as prisões e todo sistema de justiça criminal. Além disso, a história das lutas anarquistas nos informam que, em muitos casos, a luta antifascista é uma tática utilizada por liberais democratas e socialistas autoritários para conter a radicalidade do nosso anticapitalismo e de nosso antiestatismo inegociáveis. E aí chegamos a nosso ponto: somos todos, realmente, antifascistas? O que pensar de operadores das instituições de extermínio e do racismo de Estado que declaram adesão às lutas antifascistas em momentos de recrudescimento autoritário do regime político? Pensamos, especificamente, nos que se autointitulam policiais antifascistas. Ser antifascista é viver uma vida não-fascista. Como viver essa vida quando se é um agente do Estado armado e autorizado a matar? Como conceber isso? Especialmente num país como o Brasil, onde a polícia carrega toda herança escravocrata e está estruturada segundo os regimes autoritários no país durante o século XX?

Não precisamos nos aliar a mercenários armados, ensinados a obedecer sem questionar, com autorização legal para agredir e matar defendendo as desigualdades existentes em nossa sociedade. Podemos trabalhar em conjunto sob princípios de solidariedade e horizontalidade para atender às necessidades de nossas comunidades, resolver conflitos e nos defender mutuamente da violência autoritária – ou seja, da polícia, fascista ou antifascista. Não existe caminho para a liberdade que não seja através da liberdade aqui e agora. A única autonomia que construímos está nos nossos laços sociais e de solidariedade: se quisermos garantir nossa integridade física contra agressões, precisamos de redes de apoio mútuo capazes de se defender, precisamos construir autodefesa e autodeterminação, que é nossa forma de liberdade diante da abstrata e dependente ideia de segurança. Não queremos essa democracia securitária, queremos liberdade e autodeterminação: cada pessoa e comunidade agindo de acordo com sua consciência e responsabilidade coletivas, em vez da coerção inerente aos governos e aos agentes de segurança, pois estes são sempre externos aos conflitos e problemas que vida em sociedade inevitavelmente cria.

A luta antifascista deve ser aliada à luta pelo fim de todas as instituições estatais, principalmente as repressivas. Precisamos alimentar e expandir estruturas para tomada de decisão que promovam autonomia e, por fim, práticas de autodefesa que possam nos proteger daqueles que no futuro queiram se tornar nossos líderes, como nos ensinam os povos ameríndios em sua relação com as chefias. Da mesma forma que não existe luta contra opressão sem uma luta contra todo aparato policial e estatal, não existe espaço na luta antifascista para reformar uma economia capitalista, o Estado, sua polícia e suas prisões – e muito menos espaço para policiais em uma luta contra o fascismo. Se, como disse com razão um dos líderes do movimento de policiais supostamente antifascistas, a polícia é a gestão da polis, nós seremos ingovernáveis.

\chapter*{Antifa: contra o que e ao lado de quem lutar: \subtitulo{Entrevista com Mark Bray, autor do livro Manual Antifa}}
\addcontentsline{toc}{chapter}{Antifa: contra o que e ao lado de quem lutar: entrevista com Mark Bray, autor do livro \emph{Manual Antifa}}


\hfill{}\textsc{facção fictícia}

\bigskip

A luta antifascista tem atravessado cada vez mais os debates políticos, seja por meio da difusão de mensagens e ações ou das ameaças de criminalização e repressão. No início do ano, antifascistas em Porto Alegre interromperam um protesto bolsonarista em 17 de maio aos cantos de “recua, fascista“. Depois, torcedores de diferentes times de futebol se juntaram para ocupar as ruas em São Paulo e frustrar protestos de apoiadores do presidente. Ambos inspiraram ações em mais de 15 cidades, como em Belo Horizonte, Curitiba e Rio de Janeiro, onde atos estão sendo organizados semanalmente para bloquear, atrasar e impedir carreatas dos que gostam de “protestos a favor” de populistas de direita e pedindo a volta da ditadura militar.
 
Ao fim de maio, a onda de protestos combativos em centenas de cidades nos Estados Unidos, após o assassinato de George Floyd, repercutiu no mundo as lutas antirracistas e antifascistas. Como efeito, Bolsonaro e políticos da sua laia, pretendem imitar Donald Trump e tonar ações e grupos antifa em uma “ameaça terrorista doméstica”. Explicitando o que sempre dissemos: quem se incomoda e combate o antifascismo é, pela lógica, um fascista. Seu objetivo pode não ser cumprido na lei, mas podemos esperar o que sempre aconteceu na história: vai atiçar os ânimos de suas bases dispostas a praticar atos de violência nas ruas contra minorias e todos que denunciam o fascismo, com a conivência da polícia.
 
Por isso, o momento é de se organizar, nos articular e discutir sobre táticas e estratégias de luta. Assim, o coletivo Facção Fictícia convidou Mark Bray, autor do livro Manual Antifa (2019), para uma entrevista exclusiva sobre alguns temas urgentes como: relação entre movimentos antifa e black bloc, anarquismo, esquerda institucional e até os ditos policiais “antifascistas” – fenômeno até então exclusivo do Brasil e que surpreende até mesmo militantes e pesquisadores com vasta experiência nas lutas antifascistas contra todas as formas de regimes e autoritarismos.


\textbf{Como você define o que chama, em seu livro, de antifa moderna e como ela contribui para o cenário de protestos atuais contra o racismo e a polícia nos EUA?}

Em resumo, eu diria que a política ou um grupo da antifa moderna seria uma oposição militante, socialista revolucionária, orientada para a ação direta, à extrema direita que rejeita recorrer à polícia ou ao Estado para detê-los e que geralmente tem uma espécie de noção de esquerda antifascista amplamente radical (pan-radical), embora nem sempre. Como você sabe, Trump culpou antifa e anarquistas pela destruição nos recentes protestos. Embora antifas tenham, provavelmente, estado em algumas manifestações, não há evidências de sua participação nelas. Com certeza, simplesmente não há antifas suficiente nos EUA para causar tal destruição. Eu gostaria que houvesse tantos, mas não existem. Certamente, porém, antifas apoiam o Black Lives Matter e pode haver algumas pessoas que participam dos dois tipos de organização. 
 
\textbf{Quais a relações entre antifa, tática black bloc e as lutas anarquistas e anticapitalistas contemporâneas desde a emergência do movimento antiglobalização?}
 
Na maior parte, os black blocs foram usados nos Estados Unidos a partir de 1999 para protestar em cúpulas econômicas (OMC em Seattle, especialmente), protestando contra as guerras, contra as convenções políticas nacionais, etc. A associação entre antifa e black blocs nos EUA realmente começou com o J20 (protestos radicais que atacaram a cerimônia de posse de Trump em 2016), quando muitos anarquistas, antifa e outros anti-autoritários foram presos e acusados de crimes, cujas sentenças poderiam ter chegar a décadas na prisão. Felizmente eles foram absolvidos. Também ocorreram eventos como o black blocs interrompendo e acabando com um discurso do provocador de extrema-direita Milo Yiannopoulos em Berkeley, em 2017, e outros confrontos em Portland e em outros lugares. 

\textbf{Desde os levantes em 2013 e 2014 no Brail, percebemos um grande esforço das autoridades em criminalizar táticas, como os black blocs, como se estas fossem organizações formais ou criminosas, terroristas. Vemos agora, essa tentativa com antifas. Para isso, usam discursos que atacam de deslegitimam movimentos combativos, alegando que “quem pratica ações ditas ‘violentas’ (dano à propriedade, revidar a violência policial) em manifestação são ‘minorias’ ou ‘infiltrados’ para justificar o isolamento de setores radicais e a repressão estatal. Como você vê, historicamente, a resposta dos movimentos antifascistas e antirracistas a tais acusações e disputas de narrativas – que muitas vezes emergem dos próprios setores da esquerda?}
 
Bem, algumas das primeiras movimentações antifas na Alemanha da década de 1980 surgiram dos movimentos autônomos, que rejeitou basear suas políticas na aprovação da opinião pública. Portanto, nesse sentido, nem todos os antifa se importaram tanto com isso da mesma forma que outros. Mas é claro que essas disputas têm o potencial de separar os movimentos. Nas minhas entrevistas com antifascistas europeus, parece que cada movimento teve, em diferentes momentos, maior ou menor colaboração com grupos de esquerda como sindicatos etc. Tê-los como aliados pode ajudar, mas é uma aliança que  pode ser inconstante. A noção de “diversidade de táticas”, que surgiu há 20 anos ou mais durante a era do movimento  de justiça global (ou antiglobalização), foi um esforço para coexistir e contornar esses problemas. Claro que não é uma receita de bolo, depende de cada caso. Enfim, é muito difícil desfazer a dicotomia ‘bom manifestante’, ‘mau manifestante’, como fazem as imprensa e alguns grupos de esquerda. 
 
\textbf{No Brasil, nos deparamos com um fenômeno curioso, no qual policiais civis e militares se consideram “antifascitas” e se organizam enquanto movimento para se infiltrar e influenciar lutas sociais e pautas da esquerda. Em sua pesquisa, já deparou com exemplos semelhantes em outros países? Qual a sua opinião sobre a participação de policiais, militares ou outros agentes das forças de segurança estatais ou privadas em movimentos e manifestações de política radical?}
 
Isso me lembra a Europa do pós-guerra, onde todos (exceto Espanha e Portugal) estavam oficialmente do lado dos vencedores da Segunda Guerra Mundial, quando a interpretação sobre o antifascismo era simplesmente estar do lado vitorioso da guerra. Nesse contexto, houve debates tensos sobre o que significava antifascismo, especialmente porque, em países como a Alemanha ou a Itália, os “comitês antifa” socialistas revolucionários que surgiram durante a guerra, foram fechados pelos novos governos dos Aliados, de regime liberal-democrático. Os movimentos revolucionários que surgiram nas décadas seguintes, incluindo os que deram origem à antifa moderna, desafiaram a interpretação oficial do antifascismo, apontando que ainda havia muitos fascistas na sociedade e argumentando que o capitalismo oferece espaço para o fascismo. Os argumento desses grupos antifas no pós-Segunda Guerra é que o antifascismo deve ser anticapitalista. 
 
Se nós, como anticapitalistas revolucionários, permitirmos o antifascismo cair no menor denominador comum de ser literalmente “todos aqueles que se opõem ao fascismo”, perderemos essa interpretação socialista, no seu sentido mais amplo, que faz do antifascismo uma oposição enraizada na política hoje e não apenas o fato de qual lado da Segunda Guerra você estava. Então, para deixar bem claro: polícia antifa é uma puta de uma besteira. 

\textbf{As táticas antifa se mostraram a forma mais radical de resistência ao governo Trump nos EUA ao governo Bolsonaro no Brasil, trazendo uma herança de práticas radicais e anticapitalistas. Como você vê adesão da esquerda institucional, dentro dos palácios e gabinetes, aos símbolos e discursos antifa?}
 
Bem, eu acho que a criação de um movimento e um sentimento antifascista mais amplo na sociedade é importante. Idealmente, não haveria necessidade de grupos antifa específicos, porque as comunidades expulsariam os fascistas por conta própria. Como as origens do antifascismo militante podem ser encontradas na oposição a grupos fascistas e nazistas de pequeno e médio porte, faz sentido que a resistência deva ser mais ampla e maior para lidar com regimes inteiros ou grandes partidos políticos. Debato esse desafio analisando entrevistas com antifascistas que enfrentam esse impasse em um capítulo do “Manual Antifa”. Mas trabalhar em conjunto ou forjar uma coalizão não significa abandonar sua política. 
 
Esse é sempre um equilíbrio complicado: como trabalhar com aliados que não compartilham toda a sua política sem, no final das contas, realizar a agenda deles e não a sua? De uma perspectiva antiautoritária, podemos ver o que os stalinistas fizeram com os anarquistas espanhóis durante a Guerra Civil Espanhola. Este é um precedente importante a ter em mente, mas se somos fracos demais para derrotar nossos inimigos por conta própria, não podemos simplesmente concordar em ser mártires. 

Certamente, devemos criticar a cooptação institucional dos símbolos antifa, especialmente quando usados para se opor a valores centrais, como barrar a extrema direita sem recorrer à polícia ou aos tribunais (o que, obviamente, implica uma postura abolicionista penal). Talvez, em algum momento, os progressistas e moderados possam se tornar mais radicais no processo? Pelo menos nos EUA, parece que, nos últimos anos, muitos liberais, progressistas e socialistas democráticos ficaram muito mais à vontade com os socos na cara dos nazistas e, com certeza, há algo de bom nisso.

\part{Política, revolta e antipolítica antifa}

\chapter{Texto acácio/matheus prefácio adaptado do manual antifa}

\lipsum[5]

\chapter*{Antifa não é o problema: \subtitulo{A falação de Trump é uma
distração para a violência policial}}
\addcontentsline{toc}{chapter}{Antifa não é o problema: a falação de Trump é uma
distração para a violência policial}


\hfill{}\textsc{mark bray\footnote{Historiador especialista em direitos humanos, terrorismo e radicalismo político na Europa Moderna. Foi um dos organizadores do movimento \emph{Occupy Wall Street} em 2011 e seu trabalho é referência mundial no debate antifascista.}}

\bigskip

O vídeo trágico do assassinato de George Floyd pela polícia em Minneapolis te deixou com raiva? Com tristeza e desespero? Isso fez você querer queimar uma delegacia?
 
Seja esse o caso ou não (o que acho mais provável), você pode estar entre os muitos cidadãos estadunidenses que simpatizam com a explosão de raiva por trás do tombamento de viaturas policiais e da destruição das fachadas de lojas nas cidades do país após a morte de Floyd, mesmo que você não concorde com a destruição de propriedades. Embora as táticas de protesto “violentas” sejam geralmente impopulares, elas chamam atenção e nos forçam a perguntar: Como chegamos aqui?

O presidente Trump, o procurador-geral William P. Barr e seus aliados têm uma resposta simples e conveniente: “É a ANTIFA e a esquerda radical”, como Trump twittou no sábado. “Em muitos lugares”, explicou Barr, “parece que a violência é planejada, organizada e dirigida por grupos anárquicos… e extremistas de extrema esquerda usando táticas do tipo Antifa”. “Os extremistas domésticos”, twittou o senador Marco Rubio (R-Fla.), estão “aproveitando os protestos para promover sua própria agenda não relacionada ao caso”. Após outra noite de destruição que incluiu a queima do antigo mercado de escravos chamado Market House, em Fayetteville, Carolina do Norte, Trump dobrou as apostas no domingo, declarando que “os Estados Unidos da América designarão os ANTIFA como uma organização terrorista”.
 
As acusações imprudentes de Trump carecem de evidências, como a maioria de suas alegações. Mas eles também deturpam intencionalmente o movimento antifascista com interesse de deslegitimar os protestos combativos e desviar a atenção da supremacia branca e da brutalidade policial a que os protestos se opõem.
 
Abreviação de antifascista em muitas línguas, antifa (pronuncia-se “antífa—”, em português) ou antifascismo militante é uma política de autodefesa social-revolucionária aplicada ao combate à extrema-direita que remonta sua herança aos radicais que resistiram a Benito Mussolini e Adolf Hitler em Itália e Alemanha há um século. Muitos estadunidenses nunca ouviram falar de Antifa antes de antifascistas mascarados quebrarem janelas para cancelar a fala de Milo Yiannopoulos em Berkeley, Califórnia, no início de 2017 ou confrontarem supremacistas brancos em Charlottesville no final daquele ano — quando um fascista assassinou Heather Heyer e feriu muitos outros com seu carro de uma forma que assustadoramente anteviu os policiais de Nova York que jogaram suas viaturas em manifestantes no sábado no Brooklyn.

Com base em minha pesquisa em grupos antifa, acredito que é verdade que a maioria, senão todos, os membros apoiam do fundo do coração a autodefesa combativa contra a polícia e a destruição voltada contra a polícia e a propriedade capitalista que a se seguiu nesta semana. Também tenho certeza de que alguns membros de grupos antifa participaram de várias formas de resistência durante essa dramática rebelião. No entanto, é impossível determinar o número exato de pessoas que pertencem a grupos antifa porque os membros ocultam suas atividades políticas da polícia e da extrema direita e as preocupações com a infiltração e as altas expectativas de compromisso mantêm o tamanho dos grupos bastante pequeno. Basicamente, o número de anarquistas e membros de grupos antifa não chega nem perto de ser suficiente para conseguir por si mesmos uma destruição tão impressionante. Sim, a hashtag “\#IamAntifa” foi uma tendência no Twitter no domingo, sugerindo um amplo apoio à política antifascista. No entanto, existe uma diferença significativa entre pertencer a um grupo antifa organizado e apoiar suas ações online.

A declaração de Trump parece impossível de aplicar — e não apenas porque não há mecanismo para o presidente designar grupos domésticos como organizações terroristas. Embora existam grupos antifa, a própria antifa não é uma organização. Grupos antifa identificados como Rose City Antifa, em Portland, Oregon, o mais antigo grupo antifa atualmente existente no país, expõem as identidades dos nazistas locais e enfrentam a extrema direita nas ruas. Mas a própria antifa não é uma organização abrangente com uma cadeia de comando, como Trump e seus aliados têm sugerido. Em vez disso, grupos anarquistas e antifas anti-autoritários compartilham recursos e informações sobre atividades de extrema-direita através das fronteiras regionais e nacionais por meio de redes pouco unidas e relações informais de confiança e solidariedade.
 
E nos Estados Unidos, a antifa nunca matou ninguém, ao contrário de seus inimigos nos capuzes da Klan e pilotando viaturas.

Embora a tradição específica do antifascismo militante inspirada por grupos na Europa tenha chegado aos Estados Unidos no final dos anos 80 com a criação da Ação Anti-Racista, uma grande variedade de grupos negros e latinos, como os Panteras Negras e o Movimiento de Libertação Porto-Riquenho Nacional (MLN), situou sua luta em termos de antifascismo nas décadas de 1970 e 1980. Expandindo ainda mais o quadro, podemos traçar a tradição mais ampla de autodefesa coletiva contra a supremacia branca e o imperialismo, ainda mais longe através da resistência ao genocídio indígena e do legado da libertação militante negra representada por Malcolm X, Robert F. Williams, C.L.R. James, Ida B. Wells, Harriet Tubman e rebeliões de escravos. Essa tradição radical negra, feminismo negro e políticas abolicionistas mais recentes influenciadas por organizações como a Critical Resistance e Survived and Punished informam claramente as ações dos manifestantes muito mais do que a antifa (embora existam antifa negra e outras que foram influenciadas por todas as anteriores).
 
Trump está invocando o espectro da “antifa” (enquanto o governador de Minnesota, Tim Walz, culpou os “supremacistas brancos” e o “tráfico”) por quebrar a conexão entre essa onda popular de ativismo anti-racista e negro que se desenvolveu nos últimos anos e as insurreições que explodiram em todo o país nos últimos dias — que colocam a brutalidade policial em evidência, quer concordemos com a maneira como ela chegou lá ou não. Paradoxalmente, esse movimento sugere um reconhecimento não declarado da simpatia popular pelas queixas e táticas dos manifestantes: se incendiar shoppings e delegacias fosse bastante em si para deslegitimar protestos, não haveria necessidade de culpar o movimento “antifa”.
 
Esta não é a primeira vez que Trump ou outros políticos republicanos pedem que antifa seja declarado uma organização “terrorista”. Até o momento, esses pedidos não foram além da retórica — mas eles têm um potencial ameaçador. Se os grupos antifa são compostos por uma ampla gama de socialistas, anarquistas, comunistas e outros radicais, declarar a antifa como uma organização “terrorista” abriria o caminho para criminalizar e deslegitimar toda a política à esquerda de Joe Biden.
 
Mas, no caso dos protestos de George Floyd, as tentativas da direita de jogar a culpa de tudo no movimento antifa — visto por muitos como predominantemente branco — mostram um tipo de racismo que pressupõe que os negros não pudessem se organizar em uma escala tão ampla e profunda. Trump e seus aliados também têm um motivo mais específico: se as chamas e os cacos de vidro fossem simplesmente atribuídos a “antifa” ou “forasteiros” — como se alguém tivesse que viajar muito longe para protestar —, a urgência mudaria de abordar as causas profundas da morte de Floyd para descobrir como impedir o sombrio bicho-papão contra o qual Trump se opõe. Mesmo se você não concordar com a destruição de propriedades, é fácil ver a cadeia de eventos entre a morte de Floyd e os carros da polícia em chamas. A desinformação de Trump quer enganar a todos nós.



\chapter{Texto Erick/Camila (em preparação)}

\lipsum[5]

\chapter{Que fazemos do antifascismo?}

\hfill{}\textsc{bonano}

\bigskip

\epigraph{A raposa sabe muitas coisas,\\mas o porco-espinho sabe uma grande.}{Arquíloco}


O fascismo é uma palavra de oito letras que começa pela letra f. O
homem, desde sempre, foi perdidamente apaixonado pelos jogos de palavras
que, escondendo a realidade mais ou menos bem, absolvem-no da reflexão
pessoal e da decisão. Assim o símbolo age em nosso lugar e nos fornece
um álibi e uma bandeira.

Quanto ao símbolo com o qual não pretendemos nos aliar, que na verdade
nos enoja profundamente, aplicamo-lo a palavra ``anti'', consideramo-nos
do outro lado, seguros, e pensamos haver nos livrado com isso de uma boa
parte das nossas tarefas. Assim, uma vez que na mente de muitos de nós,
e quem escreve se encontra entre estes, o fascismo causa nojo, é
suficiente o recurso daquele ``anti'' para sentirmos nossa consciência
limpa, encerrados em um campo bem guardado e bem frequentado.

Entretanto a realidade se move, os anos passam e as relações de força se
modificam. Novos mestres sucedem aos antigos e o trágico bastão do poder
passa de mão em mão. Os fascistas de ontem puseram de lado as bandeiras
e as suásticas, entregues a alguns estúpidos de grandes carecas, e se
adequaram ao jogo democrático. Por que não fariam isso? Os homens do
poder são unicamente homens do poder, a balela nasce e morre, o realismo
político não. Mas nós, que de política compreendemos pouco ou nada,
indagamo-nos envergonhados pelo ocorrido, visto que nos foi retirado,
bem debaixo de nossos narizes, as antigas evidências do fascista da
camisa negra\footnote{Milícia Voluntária para a Segurança Nacional que apoiou Mussolini e o
  ajudou a dar o seu golpe de Estado, na Itália. Também estiveram
  presentes na Marcha sobre Roma, dia conhecido como o fim da democracia
  liberal italiana e a ascensão do fascismo. Essa milícia que combatia
  com violência grevistas, sindicatos e opositores do fascismo, passou a
  integrar oficialmente o exército italiano, durante o período em que
  Mussolini esteve no governo. Inspirou diversos movimentos na Europa e
  nas Américas com seus devidos uniformes prata, azul, verde etc. e até
  hoje serve de inspiração para saudosistas.} com um taco, contra os quais éramos acostumados a lutar
muito duramente. Por isso vamos à caça como galinhas sem cabeça, por um
novo bode expiatório contra o qual descarregar nosso ódio demasiado
barato, enquanto tudo em torno de nós fica mais sutil e mais esfumaçado,
enquanto o poder nos chama para discutir:

--- Mas por favor, venha para frente, diga a sua opinião, sem embaraço!
Não esqueça, estamos em uma democracia, cada um tem direitos de falar
quanto e como quiser. Os outros o escutam, concordam ou discordam e
depois o número faz o jogo final. A maioria vence e à minoria resta o
direito de voltar a discordar. Contanto que tudo se mantenha na livre
dialética de tomar os lados.

Se nós colocamos a questão do fascismo sob os termos da balela, devemos
forçosamente admitir que tudo não passou de um jogo. Talvez, uma ilusão:

--- O Mussolini, um bravo homem, de certo um grande político. Cometeu
seus erros. Mas quem não os comete? Então, deixou-se levar. Ele foi
traído. Todos fomos traídos. A mitologia fascista e antigo-romana? Deixa
disso! Ela pensa agora esse antiquário? Rouba do passado.

``Hitler... --- ironizava Klaus Mann\footnote{Escritor alemão, conhecido pelo seu romance \emph{Mephisto}, obra que
  questiona o nazismo na Alemanha, abordando a história de um ator que
  interpretava o papel de Mefisto na peça \emph{Fausto}, de Goethe, uma
  das poucas peças interpretadas durante o período. Mann era gay e fez
  parte do grupo de teatro \emph{die Pfeffermühle}, de sua irmã, Erika
  Mann, que ridicularizava os nazistas. Deixou a Alemanha com a família,
  em 1933, unindo-se a seu pai, Thomas Mann.} descrevendo muito bem a mentalidade de Gerhart
Hauptmann,\footnote{Romancista e dramaturgo alemão, ganhou o Nobel de Literatura de 1912,
  foi um dos responsáveis pela introdução do naturalismo no teatro
  alemão. Suas obras, com o decorrer do tempo, formaram um complexo
  metafísico e religioso. Era um dos artistas da \emph{lista de
  Gottbegnadeten}, uma lista elaborada por Goebbels e Hitler e continha
  os nomes de todos os artistas cruciais para a cultura nazista e que
  seriam automaticamente dispensados do serviço de guerra.} o velho teórico do realismo político --- no fim das
contas... Meus caros amigos!... Sem malícia!... Procuremos ser... Não,
se não vos importa, permitam-me... objetivos... Posso encher de novo a
taça? Esta champagne... extraordinária, realmente --- o homem Hitler,
como ia dizendo... Também a champagne, quanto a isto... Uma evolução
absolutamente extraordinária... A juventude alemã... Cerca de 7 milhões
de votos... Como disse frequentemente para meus amigos judeus... Aqueles
alemães... nação incalculável... realmente misteriosíssima... impulsos
cósmicos... Goethe... A saga dos Nibelungos...\footnote{Lendas de povos bárbaros que
  passaram por ``interpretações'' diversas ao longo da história, como a
  \emph{Saga dos volsungos} (em língua escandinava), a \emph{Canção dos
  nibelungos} (em língua alemã) e \emph{O Anel do Nibelungo} (na música
  alemã). Por meio desta, por exemplo, Richard Wagner, seu compositor,
  exaltará o povo alemão e será tomado pelo sentimento moral e cristão.
  Também a partir desta que Nietzsche se afastará de Wagner, em 1876:
  ``Eu não tolero nada ambíguo; depois que Wagner mudou-se para a
  Alemanha, ele transigiu passo a passo com tudo o que desprezo --- até
  mesmo o antissemitismo... Era de fato o momento para dizer adeus: logo
  tive a prova disto. Richard Wagner, aparentemente o mais triufante, na
  verdade um \emph{décadent} desesperado e fenecido, sucumbiu de
  repente, desamparado e alquebrado, ante a cruz cristã...'' (Nietzsche,
  \emph{O caso Wagner}).} Hitler, de um certo modo, exprime... Como eu procurei
explicar aos meus amigos judeus... tendências dinâmicas... elementares,
irresistíveis...''.

Não, sob os termos da balela não. De frente a uma boa taça de vinho, a
diferença se esfumaça e tudo se torna questão de opinião, discutível. O
belo é isso: a diferença existe, não entre fascismo e antifascismo, mas
entre quem quer e querendo-o persegue e gere o poder e quem o combate e
o refuta. Mas sob quais termos podemos encontrar um fundamento concreto
para esta diferença?

Talvez sob os termos de uma análise mais profunda? Talvez fazendo
recurso a uma análise histórica?

Não creio. Os historiadores constituem a mais útil categoria de imbecis
a serviço do poder. Acreditam saber muitas coisas, mas mais se obstinam
sobre o documento, mais não fazem outra coisa senão sublinhar a
necessidade de ser como tal, um documento que atesta de modo
incontestável o ocorrido, a prisão da vontade do indivíduo na
racionalidade do dado, a equivalência vichiana\footnote{Da filosofia de Giambattista Vico (1668 -- 1744), filósofo,
  historiador e jurista italiano.} da verdade e do fato. Cada consideração sobre possíveis
eventualidades ``outras'' redunda em simples passatempo literário. Cada
inferência, capricho absurdo. Quando o historiador tem um lampejo de
inteligência, excede imediatamente em outro lugar, nas considerações
filosóficas, e aí cai na área comum a esse gênero de reflexão. Contos de
fadas, gnomos e castelos encantados. E enquanto isso tudo ao redor do
mundo se ajusta nas mãos dos poderosos que fizeram a própria cultura dos
caderninhos de revisão, que não distinguiriam um documento de uma batata
frita. ``Se a vontade de um homem fosse livre, escreve Tolstói em
\emph{Guerra e Paz}, toda a história seria uma série de fatos
fortuitos... Se, ao invés, existe apenas uma lei que governa as ações
dos homens, não pode existir a liberdade do arbítrio, uma vez que a
vontade dos homens deve estar sujeita a esta lei''.

O fato é que os historiadores são úteis, sobretudo para fornecer
elementos de conforto. Álibis e muletas psicológicas. Quão bons foram os
federados da Comuna de 1871! Como são corajosos os mortos em Père
Lachaise!\footnote{O maior cemitério de Paris.} E o leitor inflama-se e prepara-se ele também para
morrer, se necessário, no próximo muro dos federados. Nessa espera, isto
é, à espera de que as forças sociais objetivas nos coloquem em condições
de morrer como heróis, nos arranjem a vida de todos os dias, para depois
chegarmos à soleira da morte antes que essa tão esperada ocasião nos
tenha estado à porta. As tendências históricas não são tão exatas,
década a mais, década a menos, podemos saltá-las e retornar com nada nas
mãos.

Se quer mensurar a imbecilidade de um historiador, leve-o a raciocinar
sobre a coisa em andamento e não sobre o passado. Vai ouvir algumas
belezuras.

Não, a análise histórica não. Talvez a análise política, ou
político-filosófica, como estivemos habituados a ler nesses últimos
anos. O fascismo é isso, e depois aquilo e agora isso e aquilo. A
técnica de formulação dessas análises é a seguinte. Toma-se o mecanismo
hegeliano de dizer e contradizer ao mesmo tempo, qualquer coisa similar
à crítica das armas que se torna as armas da crítica,\footnote{Referência ao ditado marxista decorrente da introdução da
  \emph{Crítica da filosofia do direito de Hegel}: ``A arma da crítica
  não pode, é claro, substituir a crítica da arma, o poder material tem
  de ser derrubado pelo poder material, mas a teoria também se torna
  força material quando se apodera das massas''.} extraindo de uma afirmação aparentemente clara tudo
aquilo que passa pela cabeça naquele momento. Você conhece o sentimento
de desilusão quando, ao perseguir um ônibus inutilmente, percebe que o
motorista, tendo o visto, acelerou ao invés de parar? Bem, neste caso
pode-se demonstrar, e Adorno me parece que o tenha feito, que é
justamente a frustração inconsciente e remota que vem à tona, causada
pela vida que foge e que não podemos agarrar, nos impelindo a desejar
matar o motorista. Mistérios da lógica hegeliana. Desse modo,
tranquilamente, o fascismo torna-se qualquer coisa menos desprezível.
Dado que dentro de nós, no recanto obscuro do instinto bestial que faz
aumentar as pulsações, fica agachado um fascista desconhecido a si
mesmo, somos levados a justificar todos os fascistas em nome do
potencial fascista que há em nós. Certo, os extremismos não! Isso nunca.
Os pobres judeus, nos fornos! Mas foram tantos assim que morreram neles?
Seriamente, pessoas dignas do máximo respeito, em nome de um mal
compreendido senso de justiça, colocaram em circulação a estupidez de
Faurisson.\footnote{Robert Farisson (1929 -- 2018), professor na universidade de Lyon, foi
  um ``negador do Holocausto'', tornando-se conhecido pelo artigo
  publicado no \emph{Le Monde} intitulado ``O problema das câmaras de
  gás ou o rumor de Auschwitz''. Farisson ``provou'' que as câmaras de
  gás nos campos de extermínio não existiam e que o genocídio realizado
  pelos nazistas não havia acontecido. Ele, assim como outros
  ``negacionostas do Holocausto'', renegam o termo e preferem ser
  chamados de ``revisionistas''.} Não, sobre essa estrada é melhor não caminhar mais.

A raposa é inteligente e portanto tem inúmeras razões próprias e tantas
outras ainda pode conceber até dar a impressão de que o pobre
porco-espinho está sem argumentos, mas não é bem assim.

A palavra é uma arma mortal. Escava por dentro o coração do homem e lhe
insinua a dúvida. Quando o conhecimento é escasso e aquelas poucas
noções que possuímos parecem dançar em um mar em tempestade, caímos
facilmente nas presas dos equívocos gerados por aqueles que são melhores
do que nós com as palavras. Para evitar casos do gênero, os marxistas,
como bons programadores da consciência alheia, em particular a do
proletariado arrebanhado, sugeriram a equivalência entre fascismo e
cassetete. Também filósofos de todo respeito, como Gentile,\footnote{Giovanni Gentile (1875 -- 1944) foi um filósofo, político e educador
  italiano. Autointitulado o filósofo do fascismo, forneceu base
  intelectual para o fascismo italiano, e escreveu, sob pseudônimo,
  \emph{A doutrina fascista}, com Mussolini.} do lado oposto (mas oposto até que ponto?), sugeriram
que o cassetete, agindo sobre a vontade, é também um meio ético, dado
que constrói a futura simbiose entre Estado e indivíduo, naquela Unidade
superior que almejam chegar tanto o ato isolado quanto o ato coletivo.
Aqui vemos, seja dito entre parênteses, como marxistas e fascistas
provêm da mesma estirpe idealista, com todas as consequências práticas
do caso: incluindo o campo de concentração. Mas caminhemos adiante. Não.
O fascismo não é somente cassetete e não é nem mesmo apenas Pound,
Céline, Mishima ou Cioran.\footnote{Autores que defenderam posicionamentos nazistas.} Não é nenhum de todos esses elementos nem ainda outros
tomados isoladamente, mas o conjunto disso tudo. Não é a rebelião de um
indivíduo isolado, que escolhe a sua luta pessoal contra os outros,
todos os outros, às vezes incluindo até mesmo o Estado, e a qual pode
também nos atrair por causa da simpatia humana que temos por todos os
rebeldes, também pelos desconfortáveis. Não, não é ele o fascismo. Não é
portanto que por defendermos sua revolta pessoal possamos titubear em
nossa visceral aversão ao fascismo. Na verdade, muitas vezes,
identificando-nos com essa defesa isolada, atraídos pelo caso da coragem
e do empenho individual, confundimos ainda mais nossas ideias e as
daqueles que nos ouvem, provocando inúteis tempestades em copos d'água.

As palavras nos matam se não prestarmos atenção.

Para o poder, o fascismo nu e cru, tal como se concretizou
historicamente em períodos históricos e em regimes ditatoriais, não é
mais um conceito político praticável. Novos instrumentos irrompem na
soleira da prática gestionária do poder. Deixemo-lo portanto aos dentes
aguçados dos historiadores, que o mastiguem conforme lhes aprouverem.
Também como injúria, ou acusação política, o fascismo está fora de moda.
Quando uma palavra é usada em tom depreciativo por aqueles que gerem o
poder, não podemos fazer uso igual. E como essa palavra e o relativo
conceito nos enojam, seria bom colocar um e outro no sótão dos horrores
da história e não pensar mais nisso.

Não pensar mais na palavra e no conceito, não no que este e aquela
significam mudando a roupagem lexical e a composição lógica. É sobre
isso que temos que continuar a refletir para nos prepararmos para agir.
Olhar hoje ao redor à caça do fascista pode ser um esporte prazeroso,
mas pode também esconder a inconsciente intenção de não querer caminhar
a fundo na realidade, por trás da densa teia de um tecido de poder que
se torna sempre mais complexa e difícil de interpretar.

Compreendo o antifascismo. Sou eu também um antifascista, mas os meus
motivos não são os mesmos de tantos outros que ouvi no passado e
continuo a ouvir ainda hoje, ao definirem-se antifascistas. Para muitos,
há vinte anos, o fascismo devia ser combatido onde estava o poder. Na
Espanha, em Portugal, na Grécia, no Chile, etc. Quando, nesses países,
ao velho regime fascista sucedeu o novo regime democrático, o
antifascismo de tantos ferocíssimos opositores se apagou. Naquele
momento, notei que meus velhos companheiros de estrada tinham um
antifascismo distinto do meu. Para mim não havia mudado muita coisa.
Aquilo que fazíamos na Grécia, na Espanha, nas colônias portuguesas e
nos outros países, poderia ser feito mesmo depois, mesmo quando o Estado
democrático havia levado a vantagem, herdando as glórias passadas do
velho fascismo. Mas nem todos estavam de acordo.

Compreendo o velho antifascista, a ``resistência'', as memórias da
montanha e todo o resto. É preciso saber escutar os velhos companheiros
que recordam suas aventuras, tragédias, os tantos mortos assassinados
pelos fascistas, a violência e todo o resto. ``Mas, ainda disse Tolstói,
o indivíduo que desempenha um papel nos acontecimentos históricos nunca
compreende seu significado. Se tenta compreendê-lo, torna-se um elemento
estéril''. Compreendo menos aqueles que, sem terem vivido essas
experiências e portanto sem se verem forçosamente prisioneiros daquelas
emoções de meio século atrás, emprestam explicações que não têm razão de
ser e que frequentemente constituem uma simples fachada para se
qualificarem.

--- Eu sou antifascista! Eles jogam na minha cara a afirmação como uma
declaração de guerra, e você?

Neste caso me vem quase sempre a espontânea resposta --- Não, eu não sou
antifascista. Não sou antifascista do mesmo modo que você. Não sou
antifascista porque combatia os fascistas em seus territórios enquanto
você estava aconchegado no calor da democrática nação italiana, a qual
nem por isso deixava de eleger o governo dos mafiosos do Scelba, do
Andreotti e do Cossiga.\footnote{Mario Scelba, Giulio Andreotti e Francesco Cossiga, políticos da
  Democracia Cristã, partido italiano fundado em 1943, por Alcide De
  Gasperi. Junto a Robert Schuman, um dos fundadores do Movimento
  Republicano Popular, partido democrata cristão francês, e Konrad
  Adenauer, presidente da União Democrata-Cristã, da Alemanha, Alcide
  fundou a Comunidade Europeia do Carvão e do Aço (CECA), a qual
  desembocou na União Europeia e que lhes rendeu o título de ``pais da
  Europa''. Visavam reconstruir uma Europa pacífica legitimando o Estado
  pós-guerra por meio da garantia do exercício das liberdades
  econômicas. O Partido Social-Democrata da Alemanha só aderirá a essa
  alternativa ao capitalismo na década de 1950, renunciando ao projeto
  de socialização dos meios de produção e defendendo o direito do Estado
  de proteger a propriedade privada.} Não sou antifascista porque continuei a combater a
democracia que havia substituído aqueles fascismos de novela, empregando
meios de repressão mais modernos e, por conseguinte, se quisermos, mais
fascistas do que o fascismo que lhe havia precedido. Não sou
antifascista porque ainda hoje procuro identificar o atual detentor do
poder e não me deixo deslumbrar pelo rótulo e pelo símbolo, enquanto
você continua a dizer-se antifascista para ter justificativa para ir às
ruas esconder-se por trás de uma faixa onde está escrito ``abaixo o
fascismo!''. Claro, se eu tivesse mais que meus oito anos à época da
``resistência'', talvez eu também estaria agora abatido pelas memórias e
antigas paixões jovens e não seria tão lúcido. Mas penso que não.
Porque, se bem se escrutinam os fatos, mesmo no amontoado confuso e
anônimo do antifascismo de formação política, havia aqueles que não se
adequavam, que se excediam, que continuavam, que insistiam indo muito
além do ``cessar fogo!''. Porque a luta, pela vida e pela morte, não se
dá somente contra o fascista de ontem ou de hoje, aquele que veste a
camisa negra, mas também e fundamentalmente contra o poder que nos
oprime, com todas as suas estruturas de sustento que o tornam possível,
também quando este poder se veste de hábitos permissíveis e tolerantes
da democracia.

--- Mas agora, podem dizer subitamente --- qualquer um poderia replicar,
pegando-me desprevenido, --- também é você um antifascista! E como
poderia ser de outro modo? É um anarquista, logo, é antifascista! Não
nos canse com as suas distinções.

E, ao invés, eu penso ser útil distinguir. O fascista jamais me agradou
e, consequentemente, o fascismo como projeto, por outros motivos, os
quais, quando aprofundados, resultam nos mesmos motivos pelos quais
jamais me agradaram o democrático, o liberal, o republicano, o
gaullista, o trabalhista, o marxista, o comunista, o socialista e todos
os outros. Contra eles, tenho oposto não tanto o meu ser anárquico, mas
o meu ser distinto e portanto anárquico. Primeiro de tudo, a minha
distinção individual, o meu modo pessoal, meu e de nenhum outro, de
entender a vida, de compreendê-la e, por conseguinte, de vivê-la, de
provar as emoções, de pesquisar, examinar, descobrir, experimentar,
amar. Dentro deste meu mundo permito o ingresso unicamente daquelas
ideias e daquelas pessoas que me agradam, do resto mantenho distância,
de maneira boa ou ruim. Não me defendo, mas ataco. Não sou um pacifista
e não espero que o nível de alerta seja ultrapassado, procuro eu tomar a
iniciativa contra todos aqueles que, mesmo potencialmente, podem
constituir um perigo para o meu modo de viver a vida. E deste modo de
vida faz parte também a necessidade, o desejo dos outros. Não dos outros
enquanto entidade metafísica, mas dos outros bem identificados, daqueles
que têm afinidade com aquele meu modo de viver e ser. E esta afinidade
não é estática, selada de uma vez por todas, mas dinâmica, se modifica e
cresce, se alarga gradualmente mais e mais, chamando outras ideias e
outros homens ao seu interior, tecendo um tecido de relações imenso e
heterogêneo, onde contudo a constante que resta é sempre aquela do meu
modo de ser e de viver, com todas as suas variações e evoluções.

Atravessei em cada direção o reino dos homens e até agora não compreendi
onde poderei pousar com satisfação a minha ânsia de conhecimento, de
distinção, de paixão perturbadora, de sonho, de amante enamorado do
amor. Em toda parte vi potencialidades imensas se deixarem esmagar pela
inaptidão e pela pouca capacidade de florescerem no solo da constância e
do empenho. Mas até onde floresce a abertura para o diferente, para a
disponibilidade de penetrar e ser penetrado, até onde não há medo do
outro, mas consciência dos próprios limites e da própria capacidade,
portanto, aceitação dos limites e da capacidade do outro, há afinidade
possível, possível sonho da façanha conjunta, duradoura, eterna, para
além de uma aproximação humana contingente.

Movendo-me para fora, para territórios sempre mais distantes daquele que
descrevi, a afinidade se devanesce e desaparece. E eis os estranhos,
aqueles que portam os próprios sentimentos como decoração, aqueles que
exibindo os músculos fazem de tudo para parecerem fascinantes. E, ainda
mais longe, os sinais do poderio, os lugares e os homens do poder, da
vitalidade cumpulsiva, da idolatria que aparenta mas não é, do incêndio
que não queima, do monólogo, da balela, do ruído, do útil que tudo
mensura e tudo pesa.

É daqui que me mantenho distante e este é o meu antifascismo.

\hfill{}\emph{Anarchismo}, nº 74, setembro de 1994


\part{Ensaio final}

\chapter{Lutas contra a extrema direita, ação direta e criminalizações}

\lipsum[5]